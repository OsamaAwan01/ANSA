\documentclass[12pt]{beamer}

\title{Week 5: Simple systems}
\usepackage{epsfig}
\usepackage{beamerthemesplit}


\begin{document}
%\maketitle
\frame{\titlepage}



\section{What is a system?}

%\frame
%{
%  \frametitle{Features of the Beamer Class}
%
%  \begin{itemize}
%  \item<1->Normal LaTeX class.
%  \item<2-> Easy overlays.
%  \item<3-> No external programs needed.      
%  \end{itemize}
%}

\frame
{
\frametitle{What is a system?}

\begin{itemize}
\item An organized structure or process that has predictable behaviour -- and
has a function. (This is a human judgement)

\item Static system (data structure) e.g. library, building.

\item Dynamic system (process + structure), e.g. computer, pendulum, human...

\item Whether static or dynamics depends on the time-scale we are interested in
describing. e.g. a library has a static organization over long times, but books
come and go in the short term.
\end{itemize}
}

\frame
{
As always we can describe as discrete or continuous.
\begin{itemize}
\item Discrete $q_1,q_2,..$ = architecture, relationships, diagrams, graphs, patterns etc.
\item Continuous $q(t)$ = variables, relationships, differential equations etc.
\end{itemize}
We have to use our imagination to come up with a description that works for us.
}


\frame
{
In the continuous picture we examine changing resources $q(t)$, it is often described
by differential equations.

In the discrete picture the sequence $q_1,q_2,...$ is described by
algorithms for transition rules, state machines, etc. This is complex except for
simple cases.

A general system consists of:
\begin{itemize}
\item Freedoms (degrees of freedom), e.g. $q,t,x,y,z,..$ the possibility for change.
\item Constraints, or rules that describe how we get from one value $q$ at one address $t,x,...$ to another $q'$.
\end{itemize}
In continuous systems it is sufficient to describe $q(t)$ and $dq/dt$. 
}

\section{Example}

\frame
{
\frametitle{Example}
Continuous, harmonic motion:
$$
\frac{d^2}{dt^2} q(t)= -\omega^2 q(t)
$$
Constraint says that the greater the distance $q(t)$ from starting point, the more we are
decelerated, until motion reverses. This is wave-like motion, pendulum, etc.
}

\frame
{
Discrete: (see last week)
$$
\hat O \vec q = \vec q'.
$$
Both these have the form 
\vspace{1cm}

``OPERATOR (SOMETHING) = SOMETHING ELSE''.
}

\section{Symmetry}

\frame
{
\frametitle{Symmetry}
\begin{itemize}
\item Sometimes it is impossible to tell the difference between different configurations
of a system, e.g. a wave has many maxima, all identical unless we have some
frame of reference to an environment.

\item Environment (boundary conditions) are thus also required to specify behaviour with all
constraints accounted for.
\end{itemize}
}

\section{Promises}

\frame
{
\frametitle{Policy, Constraint and Promises}

Policy is a collection of predecided answers to questions about
system behaviour. Components make promises about their behaviour.

Let $s$ be a promise ``sender'' and $r$ be a promise receiver. A promise
is written
$$
s \stackrel{\tau,\chi}{\longrightarrow} r
$$
where $\tau$ is the type of promise and $\chi$ is a constraint.
}

\frame
{
\frametitle{Promise valuations}
\begin{itemize}
\item Policy includes human judgement.

\item Human management can place a value on promises.

\item The value of a promise to the receiver.
$v_r \left( s \stackrel{\tau,\chi}{\longrightarrow} r\right)$

\item The receiver's Expectation that a promise will be kept.
$E_r \left( s \stackrel{\tau,\chi}{\longrightarrow} r\right)$
\end{itemize}
We can use promises to connect to economics too.
}


\section{Currencies and value systems}

\frame
{
\frametitle{Currencies and value systems}
\begin{itemize}
\item Back to dimensional analysis again.

\item Economics - we measure with money.

\item Physics - measure with energy.

\item These are all fictitious scales - what about things like ``user
satisfaction'' and ``privilege'' or ``trust''.
\end{itemize}
}


\frame
{
Other scales:
\begin{itemize}
\item Money, 
\item Security, 
\item Trust, 
\item Happiness
\end{itemize}
}


\frame
{
\begin{itemize}

\item We need relationships between these that are concretely specified if we are
to use them analytically. e.g. ``Time is money''
$$
t = m\alpha +\beta ?
$$
\item This relatioship is arbitrary -- how much is time worth? 

\item The relationship is itself POLICY.
This is also part of a system definition.
\end{itemize}
}

\frame
{
\frametitle{Summary}
\begin{itemize}
\item Freedoms vs constraints defines behaviour
\item Promise more constraints to tune behaviour as we want
\item Valuations of promises (subjective) determine whether humans
will make and keep promises.
\end{itemize}
}


\end{document}