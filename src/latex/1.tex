\documentclass{slides}

\title{Week 1: Philosophy of science}

\begin{document}
\maketitle

Rational thought began with Galileo.

System admin:
\begin{itemize}
\item Planning
\item Deployment
\item Maintenance
\end{itemize}

The course is about {\it understanding}!

Systems:
\begin{itemize}
\item Design (abstract)
\item Reliability (evaluation)
\item Efficiency
\item Integrity
\end{itemize}

Human computer systems:


\begin{itemize}
\item Unreliable / unpredictable
\item Policy - different goals
\item Predictability / Stability
\item Security
\end{itemize}

\slide{Theory and analysis:}
\begin{itemize}
\item Who are the players?
\item Interactions between them
\item Changes to system
\item Model things so we can make predictions
\end{itemize}

\slide{Science versus mathematics?}

\begin{itemize}
\item Science - observe and explain
\item Math - assume and deduce
\item Science deals with uncertainty. Cannot observe ``truth'' - always some approximation involved or interpretation (uncertainty).
\item Science guesses a model (hypothesis) - a simplified description that has the main characteristics.
\end{itemize}

\slide{Science: a continuing cycle:}

\begin{itemize}
\item Observe
\item (Hypothesize)
\item Test
\item Predict
\end{itemize}
Math (logic):

\begin{itemize}
\item Start by assuming (axioms)
\item What else is consistent with this assumption?
\item Uncover hidden truths.
\item Answer is ``right or wrong''.
\end{itemize}
Hume made a distinction between these kinds of knowledge.

\slide{Cause-effect}

For every change there is a cause that precedes it.

Obvious but important.

Not always easy to see causal chains.

\slide{Philosophy of science}

Science is based on uncertain observation and partial interpretation.
Does it represent reality?

Long history of trying to undertstand how to do science and what it means.

Need to know when we are fooling ourselves. (Marketing hype - pseudo science)

\slide{Philosphers}

Francis Bacon - look for characteristics.

Descartes/Lock  - inspired by geometry (math) and Newton's Principia.

David Hume - cannot prove true or false from observation. Two kinds of knowledge.

Immanuel Kant - our perceptions and prejudices play a role.

Karl Popper - falsification. Theory directs observation.



\slide{Technology}

Introduces a purpose or function to things -- human values.

Value judgements about how good or bad something is in relation to function.

We should use science/math to encourage reliable answers to searching questions.

Mill said science is self-correcting -- a continual process. Constant self-criticism.

Is it our ethical duty to do our best?

\slide{Abuses of science}

Seeing what we want to see.
\begin{itemize}
\item  Manipulation, suppression of data, inadequate criticism.

\item ``Scientifically proven'' ???

\item Marketing: think up ``ologies'' - shampoo and toothpaste ads.

\item Astrology -- positions of exploding gas 100 billion years ago decides our moods.
\end{itemize}
Scientists can be as irresponsible:
\begin{itemize}
\item  Power lines and mobile phones cannot hurt you! (Have to find the mechanism.)

\item Bees cannot fly!

\item Canals on Mars!

\end{itemize}
We should not be so certain we are right -- only certain that we have been
sufficiently critical.

\slide{Scales of measurement}

What is the difference between a kilometre and a kilogramme?

Arbitrary scales are initially unrealted -- give them different names.

Is there a relationship between them?

Can we have kilometres $\propto$ kilogrammes?
(e.g. Weight of fibre-optic cable).

$$
W = \alpha L.
$$
i.e.
$$
Kilogrammes = \frac{kilos}{metre} \times metres
$$
So how can we relate other quantities:e.g.
$$
Packet~per~hour \propto Packets~per~second
$$
Clearly:
$$
\frac{Packet}{hours} = \frac{Seconds}{Hours}\frac{Packets}{Seconds}
$$
$$
= 3600\times \frac{Packets}{Seconds}
$$

\end{document}