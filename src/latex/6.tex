\documentclass{slides}

\title{Week 7: A model of human-computer systems}
\usepackage{psfig}

\begin{document}
\maketitle


\slide{Synthesis...}

We should put together the building blocks we have been working on
to form a model of a human-computer system as a {\em dynamical system}.

Freedoms: Resources, flows, change, location, time, capacity for work etc..

Constraints: Network topology, hardware limitations, policy, sufficient rates of change?

We need to use currency relations, or dimensional analysis to relate the
various fuzzy observables that we used to express human needs, e.g. ``security''
and ``convenience'' etc..


\slide{Evolution of the system}

Change occurs in a system either due to development of programmed
rules or due to changes from the environment and users.

\psfig{file=loop.eps,width=14cm}

\slide{State}

We say that the state of configuration changes or evolves.

The integrity of the system means that the system should not change
outside the constraints of policy.

But...

\slide{Unpredictability}

This means that we cannot guarantee that changes will be policy
conformant, or that the system will be policy conformant at
any given moment.

We need a process of {\em maintenance} or repair.

We can only define policy of a host in relation to is {\em average}
state, over intervals of time comparable to the maintenance process.

\slide{Communication paradigm}

This is a good abstraction for discussing all kinds of changes to a system
by discrete operations.

Every change can be thought of as the enactment of a message of symbols
that represent operations. Changes might be deliberate (deterministic) or
uncontrolled random (non-deterministic).

Communication can be
\begin{itemize}
\item Autonomic: computer to computer, computer to device (disk,ram,net). 

System time is measured by the looping of these processes.

\item Human-computer: input, policy instruction

\item Human-human: collaboration, team-work etc..

\end{itemize}

\slide{Signal and noise}

We think of a system as talking to itself with a message that represents
policy:

\begin{itemize}
\item Signal: policy conformant change.
\item Noise: random change (unpredictable from environment)
\end{itemize}

\psfig{file=noisechannel.eps,width=14cm}

Error correction of ``symbols'' -- system is unreliable (Section 15.2,15.5).

\slide{Symbols}

We identify an alphabet of symbols with operations (operators).

Noise means that a policy string can be changed:

$$
\hat A\hat B\hat A\hat C \rightarrow \hat A\hat X\hat C\hat A
$$

Need to repair the $\hat X$.

e.g. $\hat B$ means \verb+chmod 644 /etc/passwd+ $\hat X$ means \verb+chmod 664 /etc/passwd+

\slide{Some notes and caveats}

The model allows us to discuss change in a common way. 

e.g. The arrival of an HTTP request causes a fault until it is repaired by
serving the page back to requester.

Tie this to service level agreements (SLA) for service or for configuration.

\slide{Cause-effect}


Faults might be complex phenomena, but solved easily by correcting a
single symbol. 

We have not identified what causes faults yet, only that if we implement
and maintain all policy symbols, faults should go away.

Have to do this regularly, or live with uncertainty. e.g. Can a security
breach remain unfixed for a whole day?

\slide{Incomplete policy}

We cannot micromanage every detail of a system, so some changes
can occur without violating policy, e.g. temporary files, ...

Management should be cheap, else it is not management, it's system redesign!

Later we shall quantify how much slack and error there is using information theory.


\slide{Arrivals}

Arrival processes / renewal processes.

We must accept randomness and uncertainty.

Can think of maintenance as a queue of service requests building up
and being serviced one by one. e.g. a help-desk

\slide{Networks}

Points of failure (use loop detection and centrality to identify)

Fault trees and networks.

Too many constraints - so a system cannot work.

\end{document}