
\documentclass{book}

\usepackage{a4}
\usepackage{epsfig}
\title{\bf\sc \Huge Analytical Network and System Administration Workbook}
\author{\sc Mark Burgess}

\usepackage{alltt}
\usepackage{times}
\usepackage{boxthm}
\newtheorem{exercise}{Exercise}
\newboxtheorem{solution}{Solution}
\usepackage{fancyheadings}
\usepackage{tocloft}
\pagestyle{fancyplain}
\lhead[\fancyplain{}{}]{\fancyplain{}{\rightmark}}
\rhead[\fancyplain{}{\leftmark}]{\fancyplain{}{}}

\def\boxx{\vcenter{\vbox{\hrule height.3pt
          \hbox{\vrule width.3pt height6pt
          \kern6pt\vrule width.3pt}\hrule height.3pt}}\;}

\def\beq{\begin{eqnarray}}
\def\eeq{\end{eqnarray}}
\def\2{\frac{1}{2}}
\def\det{{\rm det}}
\def\Tr{{\rm Tr}}
\def\rhat{{\hat{\bf r}}}
\def\i{\hat{\rm i}}
\def\ij{{\rm i}}
\def\d{{\rm d}}
\def\e{{\rm e}}
\def\union{{\cup}}
\def\intersect{{\cap}}

\def\CNOT{~{\rm\bf NOT}\,}
\def\cnot{{\neg}}
\def\CAND{~{\rm \bf AND}~}
\def\cand{\;{\cap}\;}
\def\COR{~{\rm \bf OR}~}
\def\cor{\;{\cup}\;}
\def\CXOR{~{\rm \bf XOR}~}
\def\cxor{\;{\oplus}\;}
\def\CEQUALS{~{\rm \bf EQUALS}~}

\begin{document}

\maketitle

~
\eject

\chapter*{What is the course about?}

\noindent This course is about asking searching questions about systems and answering them
as impartially as we can.
We want to ask questions like the following:
\begin{itemize}
\item How to we describe systems conceptually, separating what is important from what is not important?
\item How do we maximize performance of the system?
\item How do we minimize risk of `loss'?
\item How do we achieve stability and reliability?
\item How do we repair damage?
\item What subset of maintenance operations depends on the order in which we perform them?
\end{itemize}
To answer questions like these, we need to be more precise than we are
used to being. We need to learn to think a new language: the language of
precision thinking --- mathematics.

Some of the problems are simply artificial, designed to discipline and
train you into thinking analytically. Others are directly practical.
{\em When you are answering these questions, think of how you would
apply this knowledge, for instance in your final masters thesis.}

\vspace{1cm}
\noindent {\bf Write your solutions to these problems legibly by hand. Do not waste time typesetting the answers -- spend your time thinking about them!}

\vspace{1cm}
\noindent Some additional literature is recommended to you in answering these
problems.
\begin{itemize}
\item {\em Performance by Design}, D.A. Menasce, V.A.F. Almeida and L.W. Dowdy, Prentice Hall, ISBN 0-13-090673-5. 

\item {\em Measurement Uncertainty (Method and Applications)}, R.H. Dieck, 3rd edition, ISBN 1-55617-759-X, Intrumentation, Systems and Automation Society (ISA).
\end{itemize}

\chapter{ Philosophy of science}

The aim of this week's exercises is to make you reflect on what the course
will be about and what you hope to achieve from it.
In addition, there are some exercises that review basic analytical skills.

\begin{exercise} 
Skim through the course book to get an idea of what this course is
about. Read the introduction and the conclusions.
Think of at least one question from networking or system
administration that you would like to know the answer to, where you
think it might be possible to calculate or learn the answer by
analytical methods.
\end{exercise}
\begin{solution}
\end{solution}

\begin{exercise} 
Comment on the difference between the scientific method and the
mathematical method (logic). Can we prove that something will never
happen, or that something will always work:
\begin{enumerate}
\item Using the scientific method?
\item Using mathematics?
\end{enumerate}
What are the limitations of these approaches?
\end{exercise}
\begin{solution}
\end{solution}

\begin{exercise} 
Comment on the ethics of using the scientific method in system
administration. Is it ethical to investigate and analyse systems
relentlessly to make decisions about system policy? Would it be
just as good to `cheat' by taking short cuts and making assumptions
that are not supported by evidence? Discuss this.
\end{exercise}
\begin{solution}
\end{solution}

\begin{exercise}
Consider the statement often made in connection with security: 
``Happy users are well-behaved users''. What does this mean?
How might you go about testing this hypothesis? How could you
characterize happiness of users and well-behavedness. How might
you design an experiment to test this?
mean
\end{exercise}
\begin{solution}
\end{solution}

\begin{exercise} 
Write down and learn the upper and lower case Greek alphabets and the 
common names of the glyphs.
\end{exercise}
\begin{solution}
The Greek alphabet:

\begin{tabular}{ccc}
Alpha & A & $\alpha$\\
Beta & B & $\beta$\\
Gamma &$\Gamma$ & $\gamma$\\
Delta &$\Delta$ & $\delta$\\
Epsilon & E & $\epsilon$\\
Zeta & Z & $\zeta$\\
Eta & H & $\eta$\\
Theta &$\Theta$ & $\theta$\\
Iota & I & $\iota$\\
Kappa & K & $\kappa$\\
Lambda & $\Lambda$ & $\lambda$\\
Mu & M & $\mu$\\
Nu & N & $\nu$\\
Xi &$\Xi$ & $\xi$\\
Omicron & O & o\\
Pi &$\Pi$ & $\pi$\\
Rho & P &$\rho$\\
Sigma &$\Sigma$ &$\sigma$\\
Tau & T & $\tau$\\
Upsilon &$\Upsilon$ & $\upsilon$\\
Phi &$\Phi$ & $\phi$\\
Chi & X & $\chi$\\
Psi &$\Psi$ & $\psi$\\
Omega &$\Omega$ & $\omega$\\
\end{tabular}
\end{solution}


\begin{exercise}
Let the set $x_i$ for $i=1,2,\ldots,10$ be given by
\beq
\{ x\} = \{1,4,3,2,6,5,8,9,7,10\}
\eeq
\begin{enumerate}
\item What is $\sum_{i=1}^3 x_i$?
\item What is $\sum_{i=1}^{10} x_i$?
\item What is $\prod_{i=1}^4 x_i$?
\item What is the mean value of $x_i$, $\overline x$? Explain the meaning of the mean value of a set of numbers.
\item What is the expectation value of $x_i$, $\langle x\rangle$? Explain the meaning of the expectation value of a distribution of numbers.
\item What is the difference between the expectation value and the mean 
value of a set of values?
Given an example of usage where expectation value and
mean are the same and an example where they are different. (This question is about explaining
how an idea is used in two different ways, and it is really a philosophical question.)
\end{enumerate}
\end{exercise}
\begin{solution}
\end{solution}


\begin{exercise}
Sketch a system consisting of a web server, a DNS server and appropriate hardware,
connected to the Internet. Suppose you wanted to study and describe the following; what variables
or attributes would you want to measure?
\begin{enumerate}
\item Performance.
\item Architecture.
\item Utilization.
\end{enumerate}
In each case describe the variables you could measure or model, what values they
can assume, and explain how they 
vary or change from place to place.
\end{exercise}
\begin{solution}
\end{solution}

\begin{exercise} 
This problem is about dimensional analysis, i.e. how we relate
different scales of measurement. Dimensional analysis is a basic skill
in science and engineering, where measurements are made and relationships are
expressed.
\begin{enumerate}
\item When FastSearch left Boston, all its computers had to be transported
across the country to Sacramento. (Transportation is paid by weight.)
Suppose we  have $N$ computers and $W$ is the total weight to be
transported.
Clearly
\beq
W\propto N
\eeq
so we can write
\beq
W = kN +c,
\eeq
for some constants $k$ and $c$. If $W$ is measured in kilogrammes, what
are the units of $k$ and $c$? Since Americans do not understand SI units,
they have to convert $W$ into pounds, using a formula $P = \alpha W$.
What are the dimensions (units) of $\alpha$?

\item If $A$ is measured in nanometres and $B$ is measured in kilogrammes,
what is the meaning of $A+B$? 

\item Let $A$ be the number of arrivals (packets) per second in a computer
network. Suppose that the conversion factor $\kappa$ is a constant
measured in kilobytes per packet. Write down an expression for the
number of kilobytes that arrive in time $T$ seconds.

\item In wave theory one expresses waves in the form $A\sin(kx-\omega t)$.
Let $\theta = kx -\omega t$, where $\theta$ is an angle measured in radians,
$x$ is distance measured in metres and $t$ is time measured in seconds.
What are the dimensions of $k$ and $\omega$? The wavelength $\lambda$
is measured in metres per cycle and
is related to $k$. On dimensional grounds, argue how $\lambda$ and
$k$ must be related.

\item The frequency $f$ is measured in Hertz (cycles per second). 
How many radians are there
per wave cycle? Find a relationship between $f$ and $\omega$.

\item The speed of a wave $c$ is measured in metres per second.
Find a relationship between $f$, $\lambda$ and $c$, using
dimensional arguments.

\end{enumerate}
\end{exercise}
\begin{solution}
\end{solution}


\begin{exercise}
In a human-computer system human values are an important part of
system design.
Comment on whether you think human qualities can be quantified and
dealt with using the scientific method.
\end{exercise}
\begin{solution}
\end{solution}



\chapter{ Observations, data collection and uncertainty}

The problems this week are aimed at developing the skill of being
precise about what you say. Many misunderstandings and excuses for
poor practice in system administration are the result of lack
of adequate modelling. Mathematics helps us to formulate models in two ways:
\begin{itemize}
\item It is a precise language that is well known and standard.
\item It expresses many concepts that are relevant to systems and contains tools that will help us to unravel things that are `implicit' in our assumptions.
\end{itemize}



\begin{exercise}
This problem is about using graphs to explain relationships.
In order to estimate the average usage of a computer system a system
administrator measures the time interval between each new request for
data from disk modification times $\Delta t$ and counts the frequency
$N(\Delta t)$ with which a nearest whole number of seconds occurs.
\begin{center}
\begin{tabular}{r|r}
$\Delta t$ & $N(\Delta t)$\\
\hline
1  & 57364\\
2  & 12052\\
3  & 7005\\
5  & 4679\\
10 & 3050\\
30 & 795\\
37 & 531\\
38 & 509\\
40 & 552\\
\end{tabular}
\end{center} 
\begin{enumerate}
\item What do you think these measurements tell us about the
pattern of usage of the system?

\item Plot the data in the table:
\begin{enumerate}
\item $N(\Delta t)$ against $\Delta t$.
\item $\log N(\Delta t)$ against $\log \Delta t$.
\end{enumerate}

\item The system administrator has heard that activity that is
generated by human patterns follows an empirical law of the
form
\beq
N(\Delta t) \propto \Delta t^{-\alpha}
\eeq
for some positive constant $\alpha > 0$, and that data stored by
automated computer processes follows the form
\beq
N(\Delta t) \propto \e^{-\beta t}.
\eeq
From your data plots suggest which of these two laws best fits
the observed data and find the approximate value of $\alpha$ or $\beta$.
\end{enumerate}
Why do we care which of the above is true?
\end{exercise}
\begin{solution}
\end{solution}


\begin{exercise} 
This exercise is to remind you about the basic `hands-on' skills of calculus.
The great engineers of the Victorian age used these methods to build the
industrial revolution. You can use them to revolutionize your understanding
of systems.

When answering these exercises, approach them as you would approach
any problem in system administration: do not simply copy an answer
from somewhere uncritically, but make sure you understand all the steps
underpinning the result, i.e. create a HOW TO for solving these exercises!
The expressions below are based on expressions you will meet in the course.

\begin{enumerate}
\item Differentiate the function $\alpha(t)= c_1 t + c_2 + c_3 t^3$, with respect to $t$, where $c_i, i = 1,2,3$ are constants.
\item Differentiate the function $\beta(t)=\alpha(t)\e^{-\lambda t}$
\item Differentiate $q(t) = q_0 \left(1+\sin(2\pi t/P)\right)$ wrt $t$. Sketch
the function roughly by hand.
\item Differentiate the function $Q(t) = q_0 \left(1+\sin(2\pi t/P)\right)\e^{-\lambda t}$ with respect to $t$. Sketch the function roughly by hand.
\item Differentiate $\rho(t) = \int_0^b q(t)(t'-t)dt'$ wrt $t$.
\item Differentiate $\rho(t) = \int_0^b q(t')(t'-t)dt'$ wrt $t$.
\item Differentiate $\log(\alpha t)$ wrt $t$.
\item Find the indefinite integral with respect to $t$ of $\sin(\omega t)$
\item Find the indefinite integral with respect to $t$ of $\cos(2\pi f t)$
\item Find the indefinite integral with respect to $t$ of $\e^{-\lambda t}$
\item Find the indefinite integral with respect to $t$ of $\e^{-\lambda t}\sin(\omega t)$
\item Find the indefinite integral with respect to $t$ of $c_1 t+c_2 t^2+c_3 t^3$, where $c_i, i=1,2,3$ are constants.
\item Find the indefinite integral with respect to $t$ of $\sum_{n=1}^m c_n t^n$ where $m,c_n$ are constants.
\item Find the indefinite integral with respect to $t$ of $\log(\alpha t)$
\item Evaluate $\int_0^1\;\alpha t\,dt$.
\item Evaluate $\int_{-1}^1\;\alpha t\,dt$.
\item Evaluate $\int_0^\pi \sin(\omega t)dt$ as far as possible.
\item Evaluate $\int_{-a}^{+a} t\cos(\omega t)dt$ exactly.
\item (Hard) Evaluate $\int_{-\infty}^{+\infty} \e^{-a t^2} dt$.
\item Describe in words the fundamental theorem of calculus (hint: what is the physical interpretation of the derivative and of the integral?).
\end{enumerate}
\end{exercise}
\begin{solution}
\end{solution}

\begin{exercise} 
This exercise is about developing your use of mathematics as a natural
language for expressing ideas about systems. You need to know how to
``say'' (read/pronounce) these expressions in technical terms and in
common terms, as they express behaviours in systems.
\begin{enumerate}
\item Express $v(t) = \sin (\omega t)$ in English words.
\item Express $v'(t) = \frac{d}{dt}\sin(\omega t) = \omega \cos(\omega t)$ in Engish words.
\item Express $v'_0 =  \frac{d}{dt}\sin(\omega t)\Bigg|_{t=0}$ in English words.
\item What is the significance of an even function?
\item What is the significance an odd function?
\item What kind of function is the product of an even and an odd function?
\end{enumerate}
\end{exercise}
\begin{solution}
\end{solution}


\begin{exercise} 
This exercise is about matters related to {\em functions}. We take as
an example the use of functions of time and space to describe waves,
since the description of wavelike phenomena is one obvious
application. Some more practice at differentiation is part of the
exercise.
\begin{enumerate}
\item How many radians are there in a single cycle?
\item Sketch by hand the function $\psi(t) = A \sin(2\pi t)$ for $t \in [0,1]$.
\item Explain how this function can be considered to be a mapping.
\item What is the domain and range (co-domain) of this function mapping?
\item Suppose we drop the condition $t \in [0,1]$, what are the domain and range
of the function now?
\item Sketch (any way you like) the function $\psi(x,t) = A\sin(kx-\omega t)$.

\item The equation for a  one dimensional wave $\psi(x,t)$ is written
\beq
\frac{\partial^2\psi(x,t)}{\partial x^2} = \frac{1}{c^2} \frac{\partial^2\psi(x,t)}{\partial t^2},
\eeq
where $c$ is the speed of the travelling wave.  Show that the sine
wave
\beq
\psi(x,t) = A\sin(kx - \omega t)
\eeq
is a solution of the wave equation, provided that
\beq
k = \pm \omega/c.
\eeq
Sketch this function for some value of $k$ and $\omega$.  $k$ is
called the {\em wavenumber} and $\omega$ is called the {\em angular
frequency}.  What do these quantities represent? Use your sketch to
help you explain.

The {\em frequency} of a wave is defined by
\beq
\omega = 2\pi f
\eeq
and $f$ is measured in Hertz (Hz) or `cycles per second'. Draw one
cycle of the wave on your sketch. The wavelength of the wave is
defined by
\beq
\lambda = \frac{2\pi}{k},
\eeq
and is measured in metres (m). Draw one wavelength on your
sketch. Using the condition for the wave solution above, show that
\beq
\lambda f = c.
\eeq
 
\item Suppose we take a general linear combination of waves, with
different wave-numbers $k$ and frequencies $\omega$,
\beq
\Psi(x,t) = \int dkd\omega \; c(k,\omega)\; \sin(kx-\omega t),
\eeq
show that this combination of signals also is a solution of the wave
equation, given the same relationship between frequency and wavenumber
as before.  Hence conclude that any signal shape that can be constructed
by adding different frequencies together can only be transmitted at the speed of
waves in the carrier medium.
\end{enumerate}
\end{exercise}
\begin{solution}
\end{solution}



\begin{exercise}
Express in your own words the kinds of problems that calculus
(differentiation and integration) is useful for. Comment on the relationship
between calculus and the continuum approximation. How do these two go together?
\end{exercise}
\begin{solution}
\end{solution}


\begin{exercise}
This exercise is to remind you about the meaning of maxima and minima
of functions. Determine any maxima or minima in the following functions.
\begin{enumerate}
\item $f(x) = x-3$
\item $f(x) = (x-3)^2$
\item $f(x,y) = \sqrt{x^2+y^2}$
\item $f(x,t) = A\sin(kx-\omega t)$
\item $f(x) = k^2-x^2$.
\end{enumerate} 
\end{exercise}
\begin{solution}
\end{solution}



\chapter{From the discrete to the continuous}



\begin{exercise} 
(NEW) A system administrator's help-desk has a ticket handling system
for user queries that classifies tasks in the following states of
completion:
\begin{enumerate}
\item New problem ticket
\item Waiting to be assigned to a handler
\item Assigned to handler, waiting for response from handler
\item Reply sent to user, waiting for response from user
\item Ticket closed
\end{enumerate}
Show that this can be modelled as a finite state machine, by drawing a
diagram with appropriate transitions. Explain your diagram.
\end{exercise}
\begin{solution}
\end{solution}


\begin{exercise} 
What is meant by a finite state machine? Keyboards are configured to a
wide range of specifications around the world; all others have
additional symbols that have to be squeezed onboard.  The simplest
keyboard is the American one.  

On a Norwegian keyboard, when you press the
\verb+^+ key or
ALT GR key followed by the \verb+~+ sign, the tilde does not appear
straight until you press another key. Suggest a way of modelling this
keyboard map behaviour.
\end{exercise}
\begin{solution}
\end{solution}


\begin{exercise} 
Draw a very simple non-deterministic finite state automaton that
recognizes the TCP protocol, using the SYN, ACK, FIN flags as
transition instructions between internal states. (It does not have to
be elaborate, as long as it illusrates the main features.)

The automaton should end up in an `accept' state for valid protocol
streams. Can you make this automaton detect incorrect streams? Could
you think of a way in which this idea could be used for network
intrusion detection?
\end{exercise}
\begin{solution}
\end{solution}

\begin{exercise} 
This problem is about sets states and logic. You have to think about
how different sets of values should be grouped together, named and
identified. Consider the Venn diagram below:
\begin{figure}[ht]
\psfig{file=intersect.eps,width=4cm}
\end{figure}
Let $A$ and $B$ be abstract sets to be defined below. Decide whether
the following definitions for $A,B$ lead to a picture that agrees with the
figure.
\begin{enumerate}
\item 
\beq
A &=&  \{1,2,3,4,5,6,7\}\nonumber\\
B &=&  \{3,7,19,24\}
\eeq
\item 
\beq
A &=&  \{644,600,400,555\}\nonumber\\
B &=&  \{755,775,555,511\}
\eeq
\item 
\beq
A &=&  \{Windows,Unix,Macintosh\}\nonumber\\
B &=&  \{Debian,RedHat,SuSE\}
\eeq
\item 
\beq
A &=&  \{ 0 \le t \le 1, 2 \le t \le 3\}\nonumber\\
B &=&  \{ t=0.5, t=0.7\}
\eeq
\item 
\beq
A &=&  \{ /*/passwd, /etc/* \}\nonumber\\
B &=&  \{ /etc/passwd, /etc/shadow, /tmp/passwd \}
\eeq
\item 
\beq
A &=&  \{Windows,Linux,Solaris,MacIntosh\}\nonumber\\
B &=&  \{Sparc,Intel,AMD,68000,zSeries\}
\eeq
\end{enumerate}
\end{exercise} 


\begin{solution}
\end{solution}


\begin{exercise}
This exercise is about using sets to describe networks.
\begin{enumerate}
\item What is the set theoretical meaning of the Boolean expressions $\CAND$ and
$\COR$? Draw these as Venn diagrams. Do Venn diagrams give a good
representation of what the sets really look like, e.g. in part 4 of
the previous exercise?


\item In configuration management we think of systems as belonging to a set
of all possible system types. For instance, in cfengine systems are classified
by the sets to which they belong. e.g.
\begin{itemize}
\item {\tt Hr01} is the set consisting of all times between 01:00:00 and 01:59:59.
\item {\tt linux} means the set of all systems that run any kind of linux operating system.
\end{itemize}
Operators \verb+. & | !+ are provided to combine these using Boolean logic. The
symbols \verb+.+ and \verb+&+ have the same meaning.
\item Explain with the help of a Venn diagram showing the machines in a network
what is meant by the cfengine class expressions:
\begin{verbatim}
  i)  linux|solaris::
 ii)  Hr01.crayos::
iii)  (Hr01|Hr03).(linux|solaris)::
\end{verbatim}


\item Compare the meaning of 
{\tt linux.Hr01} and $\{ linux \} \cap \{ Hr01 \}$.
\end{enumerate}

\item Some configuration management systems use hierarchical decompositions of
the hosts in a network. Could we write the expressions above as tree-like
hierarchies? e.g. as in the figure
\begin{figure}[ht]
\psfig{file=hier.eps,width=6cm}
\end{figure}
Compare these two methods critically.
\end{exercise}
\begin{solution}
\end{solution}



\begin{exercise} 
This problem is about the use of probabilities. Probability can be used
either descriptively or predictively. When making predictions based on
past experience we must use careful judgement.
\begin{enumerate}
\item Suppose we make a survey of computers at a computing company and we count the
number (frequency) of occurrences of three categories of machine:
$S = \{ windows, linux, solaris\}$. The following results are obtained:
\begin{verbatim}
 n(windows) = 25
 n(linux)   = 16
 n(solaris) = 3
\end{verbatim}
We call this kind of data {\em statistics}. From these values, compute the
probability that a computer is windows, linux or solaris at the company.
\item Probabilities derived from data are often used to predict the likelihood
of finding a similar situation elsewhere, e.g. in intrusion detection.
Comment on whether we can use the probabilities above to estimate the
fractions of computers with these operating systems
\begin{enumerate}
\item At the same company in the future.
\item In Norway today.
\item Anywhere in the world today.
\end{enumerate}

\item Suppose that the computers counted are all desktop machines.
Given only the probabilities above, if someone tells you the size of the
company will grow to three times the total number of employees in the next year,
can you say how many windows, linux and solaris machines will be needed at the
company next year?

\item The network engineers at the company are analyzing the traffic that
passes in and out of the company to the Internet. From the packet headers they
can see the origin and destination of each packet and can therefore determine
which type of machine at the company generates which traffic. 
Using the frequency data above, what can you
say about the average level of traffic from solaris, windows and linux if
\begin{enumerate}
\item All employees work on similar tasks.
\item If employees have differentiated tasks.
\end{enumerate}

\end{enumerate}

\end{exercise}
\begin{solution}
\end{solution}



\begin{exercise}
(NEW) This problem is about using an approximation to a discrete
process to answer a practical question. We use expectation values and
integration to help us. This question addresses disk backups and many
other issues in an abstract way

The Snow Clearing Model (or Windshield Wiper Model) is a way of
thinking about maintenance in systems. Imagine that snow is falling on
the ground (many discrete snowflake events are arriving) or that rain
is falling on your car windshield. A plough or wiper-blade clears the
falling `events' and then waits. When and how should you drive your
snowplough to clear the snow? We shall show that there is a `best' answer
to this problem.

You can imagine that the snowflakes are like changes happening to the
data on a disk, or they are like problems that are occuring in a
system. We want to clear up these events by performing maintenance
(clearing the snow), but the snow never stops falling. The snow falls
mainly during the day, less at night (like users working) so we would
like to decide when the best time to start clearing is (or when should a
system administrator make a disk backup)?

To answer this, we formulate a `risk function' $\rho$ based in the idea of
Mean Time To Repair. The risk here is the risk of loss due to a
catatrophe. For instance, as long as the snow is not cleared, a driver
could crash; if there are non-backed-up changes to a disk, then the
risk of losing data is high.  Since a catatrophe can occur at any
time, the risk is proportional to the time between `now' and the next
maintenance cycle:
\beq
\rho(t_0,t_b) = \rho_0 \langle T \rangle
\eeq

\begin{enumerate}
\item Assume that maintainence (ploughing, backup) is performed once per day at time
$t_0$ and takes time $t_b$ to complete; also the intensity of events
(snowflakes, file changes) is represented by a function $q(t)$ which
is periodic (see fig. \ref{wind}). Let the length of a dayly period be $P$;
now show that we can use the function
\beq
q(t) = q_0 \left( 1 +\sin\left(\frac{2\pi t}{P}\right)\right)
\eeq
to model a Probability for Required Maintenance $p(t)$ during the
time, by normalizing. Is this always positive? Does it lie in the
range $[0,1]$?



\begin{figure}[ht]
\begin{center}
\psfig{file=wind.eps,width=8cm}
\caption{The snow clearing/windshield wiper model of maintenance. Three regions
are identified: before, during and after maintenance.\label{wind}}
\end{center}
\end{figure}

\item Suppose we can define a Time To Repair function $T(t)$ for events that occur
at time $t$, express the expectation value $\langle T\rangle$ as an integral expression
over the time period $[0,P]$

\item Suppose an event occurs in region I or in region III, what is the average time before it will be ploughed/backed up for each of these two regions?

\item If maintenance is quick, and $t_b$ is fairly short, then we can assume that
half the changes that occur in region II occur only after the plough
has made its sweep (we miss them). These changes will have to wait a
full period $P$ to be swept again. Thus the Time To Repair in this
region is the average of that in region I and $P$. What is it?

\item Show that the risk function $\rho(t)$ can be written
\beq
\rho(t_0,t_b) &=& \rho_0 \int_0^{t_0-\2t_b} q(t)(t_0-t)dt\nonumber\\
&+& \rho_0 \int_{t_0-\2t_b}^{t_0+\2t_b} q(t)(\2 P+t_0-t)dt\nonumber\\
&+& \rho_0 \int_{t_0+\2t_b}^{P} q(t)(P+t_0-t)dt.
\eeq

\item This function can be minimized with respect to $t_0$ to find the optimal time
to start ploughing. The answer is $t_0 = P/2$. Comment on this answer. Is this the time
you would intuitively pick for taking a disk backup? Explain your thoughts.

\end{enumerate}
\end{exercise}
\begin{solution}
\end{solution}



\begin{exercise}
(REVIEW) This problem is to remind you about matrix operations. Let
\beq
A &=& \left(
\begin{array}{ccc}
1 & 2 & 3\\
4 & 5 & 6\\
7 & 8 & 9\\
\end{array}
\right)\\
B &=& \left(
\begin{array}{ccc}
2 & 1 & 1\\
2 & 1 & 2\\
1 & 1 & 1\\
\end{array}
\right)
\eeq
Find:
\begin{enumerate}
\item $A+B$
\item $AB$ 
\item $BA$
\end{enumerate}
\end{exercise}
\begin{solution}
\end{solution}

\begin{exercise}
(REVIEW) This problem is to remind you about solutions to matrix equations.
\begin{enumerate}
\item Consider the matrix equation:
\beq
\left(
\begin{array}{cc}
1 & 2 \\
3 & 4\\
\end{array}
\right)
\left(
\begin{array}{c}
x\\
y\\
\end{array}
\right) = 0.
\eeq
\begin{enumerate}
\item This is really two coupled scalar equations. Write these two
equations separately. 
\item Show that $x=y=0$ is a solution of the matrix equation.
\item How can you determine quickly whether any other solutions exist? 
\end{enumerate}
\item Consider the matrix equation:
\beq
\left(
\begin{array}{cc}
4 & 2 \\
2 & 1\\
\end{array}
\right)
\left(
\begin{array}{c}
x\\
y\\
\end{array}
\right) = 0.
\eeq
\begin{enumerate}
\item Show that $x=y=0$ is a solution of the matrix equation.
\item How can you determine quickly whether any other solutions exist? 
\end{enumerate}

\item Consider the matrix equation:
\beq
\left(
\begin{array}{cc}
0 & 1 \\
1 & 0\\
\end{array}
\right)
\left(
\begin{array}{c}
x\\
y\\
\end{array}
\right) =
\left(
\begin{array}{c}
x\\
y\\
\end{array}
\right).
\eeq
\begin{enumerate}
\item Show that $x=y=0$ is a solution of the matrix equation.
\item Find any other solutions. (Hint: recall the previous problem above.)
\end{enumerate}

\end{enumerate}
\end{exercise}
\begin{solution}
\end{solution}





\chapter{Configuration management}

This exercise is about configuration management of computers and
network devices. i.e. how we construct the details of a system
and maintain them over time. It develops the abstract formulation of
the preceding exercise so as to strip away the issues surrounding
configuration management down to the essentials. By doing this, we can 
identify the fundamental issues.



\begin{exercise} 

This problem is about {\em operations} of the kind used in configuration management.

Consider the problem of file permission attributes. To keep the problem simple,
we shall consider only the attributes for the owner of a file $f$. Let the
vector
\beq
\vec P = \left(
\begin{array}{c}
1\\
r(f)\\
w(f)\\
x(f)\\
\end{array}
\right)
\eeq
be formed from a constant value 1 followed by three Boolean functions
that return 1 or 0 for the permissions for read, writing and execution of the
file.

\begin{enumerate}
\item Which of the operations below is represented by the following matrix operation?
\beq
\vec P = \left(
\begin{array}{cccc}
1 & 0 & 0 & 0\\
0 & 1 & 0 & 0\\
1 & 0 & 0 & 0\\
0 & 0 & 0 & 1\\
\end{array}
\right)
\left(
\begin{array}{c}
1\\
r\\
w\\
x\\
\end{array}
\right)
\eeq

 \begin{enumerate}
 \item chmod u+r f
 \item chmod u+w f
 \item chmod u+x f
 \end{enumerate}

\end{enumerate}


\end{exercise}
\begin{solution}
\end{solution}



\begin{exercise}

In configuration management we have a map or specification of how a
system is supposed to be configured. This map is called the {\em
policy} of the system. The process of maintenance is about checking
whether a system is still compliant with policy, and repairing it if
it isn't. Random errors and changes creep into systems for a number of
reasons. The {\em snow-clearing model} of system maintenance imagines
that random changes to the system are like falling snow and they have
to be swept away at regular intervals. The {\em immunity model} tries
to construct dumb `antibodies' (at set of operators $\hat A$) that,
when applied persistently and without intelligence, lead to the
snow-clearing model.
\begin{enumerate}
\item Let $\vec \psi$ be a vector of configuration parameters and let
$\vec \psi_0$ be the state of the vector when it complies with
policy.

The immunity model says that regardless of the current state of the system
$\psi$, some number of iterations of an operator $\hat A$ will return
the system to its policy state. Once the system has reached that
state, $\hat A$ has no further effect.
\beq
\hat A^n \psi &=& \psi_0\label{f1}\\
\hat A \psi_0 &=& \psi_0
\eeq
This property is called {\em convergence}.
Show that an operator that satisfies
\beq
\hat O^2 = \hat O
\eeq
satisfies the immunity property (i.e. it is convergent) for a state,
provided it satisfies eqn. (\ref{f1}),
and hence show that convergence is a stronger constraint than idempotence.
This property is called {\em idempotence}.

\item
Show that the matrix chmod operator is both idempotent and that it
satisfies the immunity property. 
Explain why this operator is both convergent and idempotent.
Describe or characterize the state
$\psi_0$ for the chmod operator.

\end{enumerate}
\end{exercise}
\begin{solution}
\end{solution}


\begin{exercise}
Let us extend the configuration model by allowing for the creation
and deletion of configurable objects (e.g. files). We define the
extended state vector for a file object by
\beq
\vec\psi_f = \left(
\begin{array}{c}
1\\
\sigma\\
r\\
w\\
x\\
\end{array}
\right)
\eeq
where $\sigma$ is a string that is the contents of the file and $r,w,x$
are the permission attributes as before. If $\sigma=\epsilon$ the file
is empty. If $\sigma=0$, the file does not exist.
\begin{enumerate}
\item Show that the file creation operator (like Unix {\tt creat}
or \verb+cat < /dev/null > file+
with start permissions $R,W,X$ is $\hat C(\epsilon,R,W,X)$, where:
\beq
\hat C_f(\sigma,r,w,x) = \left(
\begin{array}{ccccc}
1 & 0 & 0 & 0 & 0\\
\sigma & 0 & 0 & 0 & 0\\
r & 0 & 0 & 0 & 0\\
w & 0 & 0 & 0 & 0\\
x & 0 & 0 & 0 & 0\\
\end{array}
\right)
\eeq
\item Show that $\hat C_f$ is idempotent, i.e. $\hat C^2 = \hat C$.

\item Show that the permission setting operator (chmod) $\hat P$
is also idempotent:
\beq
\hat P_f(r',w',x') = \left(
\begin{array}{ccccc}
1 & 0 & 0 & 0 & 0\\
0 & 1 & 0 & 0 & 0\\
r' & 0 & 0 & 0 & 0\\
w' & 0 & 0 & 0 & 0\\
x' & 0 & 0 & 0 & 0\\
\end{array}
\right)
\eeq

\item Show that $\hat C_f\hat P_f = \hat C_f$ and that $\hat P_f\hat C_f \not = \hat C_f\hat P_f$, i.e. that the operators do not commute.

\item What consequences could this have for the configuration of the
system if we base configuration management on operations that do not commute?

\item What modification could we make to $\hat C$ in order to allow
$\hat C$ and $\hat P$ to commute?

\end{enumerate}
\end{exercise}
\begin{solution}
\end{solution}


\begin{exercise} Let us now extend the model above to include the editing of
file contents. This is not a trivial operation like setting a file
permission and it introduces new complexity mainly because of the
way we think about editing files -- not because there is a difference
in principle.
\begin{enumerate}

\item Consider the file modification operator $\hat M_f(\delta\sigma)$
which appends a string $\delta\sigma$ to the end of the file $f$, i.e.
such that 
\beq
\sigma\rightarrow\sigma+\delta \sigma,
\eeq
where ``$+$'' means ``join'' or append.
Show that $\hat M_f(\delta\sigma)$ can be written
\beq
\hat M_f(\delta\sigma) = \left(
\begin{array}{ccccc}
1 & 0 & 0 & 0 & 0\\
\delta\sigma & 1 & 0 & 0 & 0\\
0 & 0 & 1 & 0 & 0\\
0 & 0 & 0 & 1 & 0\\
0 & 0 & 0 & 0 & 1\\
\end{array}
\right)
\eeq
and that no other file attributes are altered by this operation.
Show that if $\hat I$ is the identity matrix, then we can write (for convenience)
\beq
\hat M_f(\alpha) = \hat I + \hat{\alpha},
\eeq
where $\alpha$ is the matrix generator of an appendage $\alpha$. Thus
we see the relative change to the file attribute vector.

\item What happens if the file does not exist before applying this operation $\hat M_f$?

\item Is the operation $\hat M_f$ idempotent?
\item Does this operation commute with $\hat C$ and $\hat P$? Do we
need to insist on any other constraints to make this true?
\item Do different modifications to the same file commute with one another?, i.e. what is:
\beq
[\hat M_f(\alpha),\hat M_f(\beta)]?
\eeq
\item Do different modifications to different files commute with one another?
\beq
[\hat M_f(\alpha),\hat M_{f'}(\beta)]?
\eeq

\item In cfengine we have the operator {\tt AppendIfNoSuchLine}
which can be interpreted as a conditionally constrained $\hat M$ (call it
$\hat m$).
This operator has the form
\beq
\hat m_f =\left\{
\begin{array}{ll}
\hat I+\hat{\delta\sigma} & \mbox{if $\delta\sigma \not\subset\sigma$}\\
\hat I  & \mbox{if $\delta\sigma \subset \sigma$}.
\end{array}
\right.
\eeq
where $\sigma$ is understood to be the current contents of the file
prior to operation. Is this operation idempotent? Is it convergent?

\end{enumerate}
\end{exercise}
\begin{solution}
\end{solution}

\begin{exercise}
This exercise summarizes the foregoing exercises.
\begin{enumerate}
\item If you were writing a new tool for automatically configuring
and maintaining devices with special properties what properties would you 
need to secure of this tool?

\item What is the difference between the properties of idempotence and
convergence?

\item In the configuration management tool cfengine, operators are
purposely made to be convergent wherever possible. The operators
can also be made to commute to avoid ordering problems.
Some authors have proposed an alternative to having many commuting
operators by suggesting that any maintenance should begin by
reinstalling the system from a fixed image and then applying configuration
changes in a strict order to guarantee the outcome. They call this
approach {\em congruence}, i.e. the correctness of policy configuration
depends on the precise order in which the operations are performed from
a known state.
\begin{itemize}
\item Convergence: $\hat C(\sigma), \hat P(r,w,x),\hat m(\delta\sigma), \ldots$ (Many operators)
\item Congruence: $\hat T(\sigma+\delta\sigma,r,w,x,....)$ (One big operator)
\end{itemize}
\begin{enumerate}
\item Are these alternatives both idempotent?
\item Is either one of these methods more reliable than the other? 
\item Is either one
more correct than the other? 
\item Can both of these methods be used
`on the fly' while devices are in operation?
\item Given that regular maintenance is necessary, which of these approaches
does the least violence to the system?

\item The Simple Network Management Protocol (SNMP) which is used for
monitoring and configuring simple network devices has two operations:
GET and PUT (like READ/WROTE or PEEK and POKE). Are these operations idempotent?
Is there any problem with using SNMP to configure workstations and
computers?
\end{enumerate}
\end{enumerate}
\end{exercise}
\begin{solution}
\end{solution}


\chapter{ Simple systems}



\begin{exercise} 
This question is about the fundamentals of systems: what constitutes a system and what
makes one system better than another.
\begin{enumerate}
\item Explain the difference between a static and a dynamic system.
\item Describe the necessary components of a dynamical system.
\item What is meant by freedom and constraint? Identify the freedoms and constraints
in the following systems:
\begin{enumerate}
\item A pendulum.
\item Web client-server communication.
\item A help-desk.
\item A configuration engine (e.g. cfengine).
\item An SNMP monitor (e.g. MRTG).
\end{enumerate}

\item Explain the relationship between an algorithm and a protocol, and explain
how freedoms and constraints enter into these.

\end{enumerate}
\end{exercise}
\begin{solution}
\end{solution}


\begin{exercise} \label{busopp}
(NEW) In business modelling, one begins often with an activity chart,
which is a crude finite state representation of a customer's behaviour
within a system. This allows us to measure user behaviour as a pretext to
capacity planning. Examine fig. \ref{bus}
\begin{figure}[ht]
\psfig{file=business.eps,width=12cm}
\caption{An activity diagram for customers arriving at an online
bank's website. Transition probabilities between the different activities
are shown on arrows. Each state has a link to the exit state (x),
which is not drawn.\label{bus}}
\end{figure}

\begin{enumerate}
\item Write down the transition matrix for this set of states $\{e,h,s,v,g,c,b,x\}$.
\end{enumerate}
\end{exercise}
\begin{solution}
\end{solution}

\begin{exercise}
(NEW) This problem is about a technique called Linear Programming. It
is a simple technique, in two dimensions that used to be taught to
school children. It teaches you to use a pen and paper to solve simple
constraints.

A service offering business runs two different services,
$X$ and $Y$, for its customers; these have service rates of $x$ and $y$
Gigabits per second.  It leases a network connection with a maximum
capacity of $C$ Gigabits per second. The company pays $R$ euros per
second for its network connection, so it has to earn at least this
much to make a profit. We want to find out how to choose (prioritize)
$x$ and $y$ so that we can program a DiffServ router to maximize
the profit for the company.

Consider fig. \ref{linear}. We do this by taking the contraints we know
and drawing them as lines and excluded regions on a diagram like this.
For instance, we know that
\beq
x + y \le C
\eeq
since the service levels cannot be larger than the transport limit. To
represent this on the disgram, we draw the line $x+y = C$ and shade the
region above it to show that it is excluded.

\begin{figure}[ht]
\begin{center}
\psfig{file=linearprog.eps,width=6cm}
\caption{Linear programming uses a diagram like this.\label{linear}}
\end{center}
\end{figure}


\begin{enumerate}
\item Draw the axes and exclude the region $x+y\le C$ as explained above.

\item For dependency reasons, the company knows that the traffic to Y
will be at least twice that to X, thus $y \ge 2x$. Plot the line $y=2x$
on your diagram and shade the region under it.

The clear (unshaded) region in this area represents the company's 
``business plan'' or operating zone. As long as $x$ and $y$ are in this
area, or on its boundary, the constraints are obeyed.

\item Let $P_x,P_y$ be the price per completed transaction (euro/tr), let
$D_x,D_y$ be the data per transaction (Gb/tr). The, using dimensional
analysis, show that the rates of earning/income of each machine (euros/sec) 
are $P_x x/D_x$ and $p_y y/D_y$.

\item The company knows that it can charge up to three times as much for
X as for Y, i.e. $P_x \le 3P_y$, so they want to maximize traffic stream $x$.
Unfortuantely they can only limit the maximum level using the Diffserv router.
If they limit $y$ to allow more $x$ in, they do not know that the demand for
$x$ will fill the empty capacity. They must choose the price and maximum
values of the streams to maximize their profits. Can they calculate
a best answer?

\item What is the condition for the company to make a profit?

\end{enumerate}
\end{exercise}
\begin{solution}
\end{solution}




\begin{exercise} 
This exercise is about system architectures or structural properties
of systems.
\begin{enumerate}
\item Compare the structure of a StarLAN network with the architecture of a
Linux Beowulf cluster. What are the similarities and differences? (Discuss this in terms
of the main elements of a system.) When would you choose these architectures, i.e. what
tasks are they best suited to?

\item What is meant by top-down and bottom-up design of a system? Comment on the use of
these strategies for (a) system design and (b) system maintenance.

\item What is meant by hierarchical organization? 
Explain how the depth of a system hierarchy can affect the
performance of a system.

\item Abstraction is used in system design to separate independent issues from one another.
Explain how a hierarchy achieves such a separation. Is the hierarchy the only
model of separation of independent issues?

\item What is meant by system normalization? Explain the purpose of normalization.
\end{enumerate}
\end{exercise}
\begin{solution}
\end{solution}

\begin{exercise} 
This exercise is about system architectures or structural properties.
\begin{enumerate}
\item What is meant by a data structure? What is its relationship to systems?
\item Explain how the organization of a system can affect the
efficiency with which different types of input are processed. Give
an example of a system that is designed to handle a special kind of
input.
\item How would one view a network design as a data structure? Explain how the use of
an abstract data model can help in the design of a network.

\item Compare an old-fashioned shared-bus Ethernet architecture (or Hub) with a StarLAN switched
architecture and a wireless WLAN network. Assuming that a number of
PCs are connected together by these two methods, comment on the
freedoms and constraints in the two systems. Comment on the criteria
you might use for determining which type of design would be best in a
given situation.
\end{enumerate}
\end{exercise}
\begin{solution}
\end{solution}


\chapter{Reading week}

You should use this time to do some general reading in the book to
enlarge your knowledge of the course materials and work through
examples.

\chapter{Diagrams and graphs}


\begin{exercise} 
(NEW) This exercise is about thinking of structural problems in terms
of graphs. Graphs are a powerful way of analysing many qualitative and
quantitative relationships.
\begin{enumerate}
\item A company has three clients for which it provides a service. Draw this relationship as a graph.
\item The same company decides to outsource part of its business to another company. Modify your diagram to show this.
\item After a while, the risk of relying on a single outsourcing company is found to be unacceptable and outsourcing is divided between two separate companies. Modify your graph to show this.
\item Comment on how you might represent load-balancing in a graph, using weighted links. 
\end{enumerate}
\end{exercise}
\begin{solution}
\end{solution}


\begin{exercise} 
Encryption keys for private communication between users or systems are
of two kinds: (i) shared keys (128 bytes), or (ii) public-private key
pairs (1024 bytes + 8 bytes).
\begin{enumerate}
\item In order to communicate privately with (i), any pair of users must share
a unique key. Show that if there are $N$ users who need to communicate potentially
with every other user, a total of $N(N-1)/2$ keys is required.

\item To communicate privately with public keys, show that each user needs only 2 keys,
i.e. the total number of keys is $2N$.

\item Calculate how many users (find $N$) such that it is worthwhile introducing
public-private keys, i.e. before the number of shared keys is much
larger than the number of public-private keys.

\item Calculate how many users (find $N$) such that it is worthwhile introducing
public-private keys, i.e. before the amount of memory used to store
the shared keys is much larger than that required to store the of
public-private keys.

\item Comment on which of the two criteria above makes most sense.
\end{enumerate}
\end{exercise}
\begin{solution}
\end{solution}


\begin{exercise} 
(NEW) Find the centrality eigenvector for the transition matrix
in exercise \ref{busopp}.
\end{exercise}
\begin{solution}
\end{solution}


\begin{exercise}
This exercise is about determining the relative importance of nodes in
a network.
\begin{enumerate}
\item Consider the graph:
\begin{figure}[ht]
\psfig{file=trivial.eps,width=4cm}
\end{figure}
Show that the graph has adjacency matrix
\beq
A = \left(
\begin{array}{ccc}
0 & 1 & 0\\
1 & 0 & 1\\
0 & 1 & 0\\
\end{array}
\right)
\eeq

\item Find, by hand, the eigenvalues $\lambda_i$ and 
eigenvectors $\psi_i$ of this matrix, satisfying
the secular equation
\beq
A \vec \psi_i = \lambda_i \psi_i.
\eeq
The principal eigenvalue is special; 
explain how the components of this vector characterize the graph.

\item Consider now a metropolitan area network of routers in the figure:
\begin{figure}[ht]
\psfig{file=nontrivial.eps,width=6cm}
\end{figure}
Find the adjacency matrix of this network and find its principal eigenvector.

\item Assuming that all of the routers are connected to local networks
that are approximately as active as all the others, rank the nodes
according to their importance as communication hubs in the network.

\item The remainder of this problem is about the interpretation of the
eigenvector ranking. Explain how the local importance $I_i$ of each node $i$
is proportional to the number of nearest neighbours that interface with it:
\beq
I_j &=& \sum_{i = {\rm neighbours~ of}~ j} 1\nonumber\\
    &=& A \vec 1,
\eeq
where $\vec 1^{\rm T} = (1,1,1,...,1)$. Hence show that a weighted sum
of neighbours with weights $\alpha_i$ for the $i$th neighbour, i.e.
\beq
\vec I = A \vec \alpha.
\eeq
Show that, if we weight the local importance of each node
with the relative importance of its neighbours,
i.e. $\vec \alpha \propto \vec I$, then we find the following self-consistent
equation for $\vec I$:
\beq
A \vec I = \lambda \vec I,
\eeq
for some constant of proportionality $\lambda^{-1}$. Hence one
concludes that $\vec I$ is both the importance ranking over all the nodes
and the solution of the eigenvector equation, i.e. that the two are equivalent.

\item How useful do you think this ranking is? What assumptions lie behind it?

\item (Hard) The eigenvector equation has $N$ solutions if there are $N$ nodes.
Can you see why the solution we want is the principal eigenvector?
\end{enumerate}
\end{exercise}
\begin{solution}
\end{solution}





\chapter{ A model of human-computer systems}

\begin{exercise} 
An antarctic research station measures that the signal strength of
its satellite link varies during the course of a day according
to the empirical formula
\beq
S(t) = S_0(1+\sin(2\pi t/24))
\eeq
where time is measured in hours GMT and $t=0$ is calibrated to 5:30 a.m GMT.

The system administrator guesses that the rate of transfer of data 
$R(t)$ via its satellite line varies with time in strict proportion
with the signal strength: $R(t) \propto S(t)$.
\begin{enumerate}
\item Does the assumption that data rate is proportional to signal strength
sound reasonable to you? Explain your answer.

\item Based on his assumption, the system administrator scribbles down
a formula for the data rate in {\em Giga-bytes per second}.
\beq
R(t) = R_0 (1+\sin(2\pi t/24)).
\eeq
If the maximum data rate is known to be 1 Giga-bit per second,
find the value of $R_0$.

\item To be certain of being able to download the evening's movie entertainment
in time for its screening, the administrator needs to calculate how
long the data transfer will take. He writes down the equation
\beq
d(t) = \int_{t_s}^{t_f} R(t')dt'
\eeq
for the amount of data $d(t)$ (in Giga-bytes) received as a function of the start time
$t_s$ and final time $t_f$, on the cailbrated hour-scale, given that
he starts downloading at time $t_s$. Explain the thinking behind this
formula and explain why the system administrator has made a simple
mistake, which has to do with measuring units.

\item The administrator corrects his mistake and starts with
\beq
d(t) = 3600 \int_{t_s}^{t_f} R(t')dt'
\eeq
Suppose that the movie must be downloaded by 18:00 GMT; by solving
this equation for $t_s$, calculate how long it takes to download 4
Giga-bytes of data, and then 40 Giga-bytes of data. (Hint: be careful
to convert times to the calibrated scale.)

\item As more penguins start to run Linux, the total bandwidth available
to the station has to be shared amongst mulitple customers and the
station can only use one tenth of the total channel capacity. Recalculate
the time taken for the download if the total channel capacity is reduced to
i) $1/10$ and ii) $1/100$ of the total capacity. How would the answers
be different if the film was to be downloaded for 21:00 GMT.

\end{enumerate}
\end{exercise}
\begin{solution}
\end{solution}

\begin{exercise} 
A system administrator is trying to decide when she should take a backup of
the system server. By measuring the number of client connections to the server
over a number of weeks, she comes up with the following formula for the {\rm average}
number of connections to the server.
\beq
N(t) = 10^4 \left(\frac{3}{2}+\sin\left(
\frac{2\pi t}{24}+\phi
\right)\right)
\eeq
where $\phi$ is a constant phase angle which she will use to fit the curve to
her observations.
\begin{enumerate}
\item The number of connections is an integer, but the sine function is a continuous
function. Explain why this makes sense for the average number of connections.

\item The maximum download activity is observed to occur in the middle of the
day at 12:00 noon ($t=12$). By differentiation, find the maxima and minima of
$N(t)$ and determine the value of $\phi$ such that the maximum value of
$N(t)$ is at 12 noon.

\item The system administrator decides to take a backup of the system when there
is least activity. Suggest a time to take a
backup, based on the formula.

\end{enumerate}
\end{exercise}
\begin{solution}
\end{solution}


\begin{exercise} 
This exercise is about arbitrary decisions made in deciding policy.
In modelling human-computer systems we are often required to make
value judgements about different resources. In order to compare the
different scales or `currencies' we must relate them to one another. This allows
us to make trade-offs and decisions about strategy. There are no right
answer here; this problem is meant to illustrate how difficult it is
to be precise about human-motivated policy.
\begin{enumerate}

\item What is meant by the expression `time is money'? Can you formalize this expression mathematically?
\item Suppose that knowledge/experience is represented in a database in  a binary tree data structure. This
knowledge is used as a cache to avoid costly computation (e.g. a web browser, a DNS resolver or a numerical
processor). Explain how you would go about describing the amount of CPU/network time saved as a function
of the size of the cache.
\item Suggest a formalization of the expression `security is the opposite of convenience'. Check that
you answer makes sense. Criticize your expression, noting any problems or limitations.
\item The convenience of a having all hosts in a network configured identically is often used to express
an amount of time saved in setting up the machines. Suggest an expression for the convenience associated
with having all hosts identical. Does this ``agree'' with your expression about security above?

\item Is convenience for users the same as convenience for system administrators? Suggest a relationship
between convenience for users and convenience for system administrators for some company of your choice.
Can you think of an example of a company where the relationship is very different from your
first example?
\end{enumerate}
\end{exercise}
\begin{solution}
\end{solution}

\begin{exercise} 
This problem is about grading or ranking system qualities using arbitrary currencies as
measures of integrity. This is a common practice in computer security.
\begin{enumerate}
\item In the Biba model of computer integrity one uses integrity levels as a currency
for data and system security.  If data or parts of a system are
contaminated by contact with something with a lower integrity level
then the integrity level of the data must be downgraded.

Consider a firewall model in which there is a ``secure'' region on the inside of the firewall
and an ``insecure'' region outside. Users and systems need to access
data on the inside of the firewall from outside. By assumption the
integrity level outside is lower than that inside.
\begin{enumerate}
\item If users or hosts outside the firewall can read information from the secure area, is
the integrity of the information downgraded outside the firewall?

\item If users or hosts outside the firewall can write information to the secure area, is
the integrity of the information downgraded inside the firewall?
\item Explain the usefulness and limitations of a firewall, especially when used in
conjunction with a Virtual Private Network that tunnels through it?
\end{enumerate}
\end{enumerate}
\end{exercise}
\begin{solution}
\end{solution}




\chapter{Integrity, information and noise}

\begin{exercise}
This problem is about the technical meaning of informational entropy and how it
applies to noisy measurements.
\begin{enumerate}
\item Explain what is meant by the uncertainty of an observation or measurement.
\item How is the Shannon entropy defined for a data stream with $C$ possible measurement classes?
\item How do these classes of measurement relate the to information in a coded message?

\item Explain what is meant by a high entropy message and a low entropy message.

\item Suggest two different ways in which $C$ symbols can be coded for transmission
by optical fibre.

\item The usual model of signal noise is to assume that the true signal can be separated
from a background of noise that we know little about. It is assumes that the
sources of noise are independent, i.e. that they make orthogonal contributions to the
uncertainty of signal measurement,
\beq
{\rm Error}^2 = \Delta q_1^2 + \Delta q_2^2 + \ldots + \Delta q_C^2.
\eeq
Explain the reasoning behind this formula.

\item One way to model this noise is to say that we wish to assume as little as
possible about it, i.e. we have maximal uncertainty but we know that it has
a limited extent. This leads us to a model of Gaussian noise.

Use the method of Lagrange multipliers to find the noisiest
probability distribution $p_i$ of symbol errors $\Delta q_i = q_i -
\overline q_i$, subject to the condition that the noise falls within a
bandwidth of approximately $\pm\sigma$ of the true signal $\overline
q_i$. i.e.  Maximize the Lagrangian function
\beq
L = -\sum_{i=1}^C p_i\ln p_i - \alpha\left(\sum_{i=1}^C p_i -1\right)
-\beta \left(\sum_{i=1}^C p_i (q_i - q_i)^2/\sigma^2 -1\right)
\eeq
Hence show that the distribution that arises from many independent sources of noise
is the Gaussian (normal) distribution.

\item Consider the continuum limit of the measurement classes, i.e. $C\rightarrow \infty$,
where the summation over $i$ is replaced by an integral over a range of $q$.
Use the argument in section 15.6 to derive the well-known Shannon formula for the capacity
of a noisy channel:
\beq
C = B \log \left(1+\frac{S}{N}\right).
\eeq
where $S$ is the signal power, $N$ is the noise power and $B$ is the bandwidth of
the signal classes.
\end{enumerate}
\end{exercise}
\begin{solution}
\end{solution}

\begin{exercise}
(Difficult) This exercise is about the interpretation of entropy to system integrity.
\begin{enumerate}

\item Show that the Shannon entropy $H$, multiplied by $N$, can be interpreted
as the average length of the label that is needed to uniquely
distinguish a message from all others of length $N$, within a fixed
alphabet, i.e. that it .

In other words, this says that $NH$ tells us the shortest number of
symbols in which we can describe the exact message assuming one has an
appropriate coding scheme that maps messages into strings
of the specified alphabet. Hence it is the
amount of non-redundant information in the message.


\item In computer security, one often speaks about the entropy of an
encryption key. What is meant by this terminology?

\item In what sense is system policy a message that is to be transmitted?

\item Suggest a simple alphabetic reclassification to compress the message:
\begin{verbatim}
BIGCATBIGCATBIGCAT
\end{verbatim}
by a factor of 3.

\item The entropy (log base $m$) represents the uncertainty per message symbol, in units of `digits belonging to an alphabet of $m$' symbols.
Calculate the $\log_6$ entropy of the message above. What is the
uncertainty per symbol of the message? Explain what this means.

\item If the entropy $H$ can tell us about the compression of a message,
what does this tell us about the last example?  
Take the $log_6$ entropy and multiply it by the length of the message
above. Is the result shorter than the original message above? (Explain
your result.)

\item Entropy assumes that the order of symbols in a string 
must be distinguished
to capture the information in a message. Suppose we
form an alphabet of $C$ commuting, convergent operators. What is the
shortest number of symbols that can faithfully transmit {\em any}
message from this alphabet to an unconfigured computer?

What is the total number of different messages one can create from
$C$ symbols
\begin{itemize}
\item If order matters?
\item If order and repetition  do not matter?
\end{itemize}

\item Consider a system in which there are eight possible policy rules that
can be violated. The faults are called $\Delta_1,\Delta_2,...\Delta_8$.
Over time, a statistical observation tool (like cfenvd) observes that
these faults occur with the following frequencies:
$$
p(\Delta) = \left(\frac{1}{2},\frac{1}{4},\frac{1}{8},\frac{1}{16},\frac{1}{64},\frac{1}{64},\frac{1}{64},\frac{1}{64} \right)
$$
in strings of the form:
$$
\Delta_1\Delta_2\Delta_1\Delta_3...
$$
The mean time before failure is $\langle T\rangle$ (or average time
between $\Delta_i$).

Consider a set of operations that counters each of these problems
$\hat O_1,\hat O_2, ...$. It is rarely possible to detect change and
apply these until a certain time has elapsed, so imagine that several
faults can have occurred in the time it takes to
schedule maintenance (Mean Time To Repair or MTTR). Faults that occur
most frequently should be performed most often, but multiple repairs
can be eliminated when performing scheduled (not event driven)
maintenance.  This allows us to compress the work to be done by
avoiding multiple operations of the same type, at the expense of the
prolonged uncertainty.

If $T_m$ is the time between maintenance checks, show that the average
number of faults to be corrected is the uncertainty of the system over
the time: $$ {\rm Maintenance-length} = H\langle T_m/T\rangle.  $$
This is shorter than the full average fault number $N =\langle
T_m/T\rangle$ of full symbols.  Explain why.

Based on the previous questions, can these information-compressed
maintenance operations be shorter than a maintenance procedure based
on commuting, convergent operations? 

\end{enumerate}
\end{exercise}
\begin{solution}
\end{solution}




\chapter{Arrivals and queues}

\begin{exercise} 
Explain what is meant by a time-series.
Explain what is meant by the Hurst exponent for a time-series. What does it show?
\end{exercise}
\begin{solution}
\end{solution}


\begin{exercise} 
This question is about task management and service handling.
It concerns the efficiency of handling services by queueing, and the
deployment of resources in parallel to cope with demand.
\begin{enumerate}
\item Explain what is meant by the term ``arrival process''. What is the most
common model for such a process?

\item Consider the arrival of queries to an ISP help desk. The arrival process
contains both long and short jobs in random order. How would you
determine the type of the arrival process? What do you think is the
best way to model such a process to discuss the efficiency of the
queueing?

\item Two ISPs compete for customers by claiming to have an efficient help service.
The two companies decide to organize their services differently (see
figure).
\begin{figure}[ht]
\psfig{file=queues.eps,width=8cm}
\end{figure}
Both ISPs have six persons staffing their phone lines. ISPA decides to
use a telephone menu to divide queries into different types in
advance, so that a specialist can deal with each problem. ISPB decides
that it is best to keep all lines open for all queries.

Use you intuition to argue which of these two methods is most
efficient. Hint: what happens if there are just three calls in the
queue all on the same specialist subject?

\item ISPA and ISPB both receive an average of
0.9 calls per minute. The average service time at both companies is 5
minutes per call.

Assume the arrival processes are Poissonian in nature.  Explain how
you would compare the efficiency of these two queueing methods
formally.  Which system is superior?

\item Calculate the utilizations of the help desks of ISPA and ISPB.

\item Suppose that arrivals follow a long-tailed distribution for
inter-arrival times (non-Poissonian).
What does this mean? How might it affect the results?

\item Compare the conclusions you have found in this exercise with the 
folk theorems for redundancy in chapter 18.
\end{enumerate}
\end{exercise}
\begin{solution}
\end{solution}


\chapter{Workflow models}

\begin{exercise} 
This exercise is about the scalability of system management. It relates to the models
described in section 18.5-18.6.

Suppose errors occur in $N$ devices in the manner of a random arrival process that
is equally likely to affect all devices (hosts, routers etc) in a network.
\begin{enumerate}
\item If the average failure rate for the total network is $I$ faults per second, what
is the average failure rate {\em per device}.
\item Suppose that we centralize the administration of all the nodes so that a system
administrator or automatic system (like an SNMP monitor) deals with all the errors that occur.
Suppose that this centralized administrator has a work capacity of $C$ repairs per second,
how much work can be done on each host on average?

\item We call the behaviour of the expressions as a function of the number of devices $N$
the scaling behaviour of the model. Explain what is ``good'' scaling behaviour
and what is ``bad'' scaling behaviour in a maintenance model.

\item Explain the expression net failure rate of systems over time in eqn. (18.55).
What does this expression really mean? What are its limitations?

\item Suppose the probability of the central controller being able to communicate with a device
is less than 1, i.e. communications are unreliable. How does this affect the scalability of the
model?

\item Suppose that every device in a network receives its policy from a central
location, but is able to carry out repairs by itself. What is the scalability of this
model?
\end{enumerate}
\end{exercise} 
\begin{solution}
\end{solution}


\begin{exercise} 
Consider the directory service LDAP. 
\begin{enumerate}
\item What function does it perform?

\item Discuss this system from the perspective of the basic system principles.
Mention redundancy, hierarchy, dependency. 

\item What principles are at
work in LDAP? 

\item If you were to verify how effective this software is
at performing its task, how would you test this by experiment?
\end{enumerate}
\end{exercise}
\begin{solution}
\end{solution}


\begin{exercise}
This problem relates maximization of resources to graphical methods of analysis-

The network connectivity of $N$ hosts can be written
\beq
\chi = \frac{1}{N(N-1)} h^{\rm T} A h
\eeq
has a maximum value when all hosts are connected and are available for
communication. Consider a pervasive computing network in which the available
channels of communication are described by the adjacency matrix $A$.
Use the method of Lagrange multipliers to maximize $\chi$ with respect
 to $h^{\rm T}$ subject to the condition $h^{\rm T} h = H$, i.e. the
total availability of resources in the network is fixed.
Show that $\chi$ is maximized when
\beq
A \vec h = \lambda \vec h,
\eeq
for some constant $\lambda$.

Use Octave, Mathematica or some other tool to find $\hat h$ and $\chi$
for the network:
\begin{figure}[ht]
\psfig{file=connect.eps,width=5cm}
\end{figure}
i.e.
\beq
A = \left(
\begin{array}{ccccccc}
0 & 0 & 1 & 0 & 0 & 0 & 0\\
0 & 0 & 1 & 0 & 0 & 0 & 0\\
1 & 1 & 0 & 1 & 1 & 0 & 0\\
0 & 0 & 1 & 0 & 0 & 0 & 0\\
0 & 0 & 1 & 0 & 0 & 1 & 1\\
0 & 0 & 0 & 0 & 1 & 0 & 0\\
0 & 0 & 0 & 0 & 1 & 0 & 0\\
\end{array}
\right).
\eeq
Which node in this network needs to be able to handle to greatest traffic load?

\end{exercise}
\begin{solution}
\end{solution}




\begin{exercise} 
(Difficult) This problem is about algebraic modelling collaboration
and dependency in systems. The aim here is to see how simple
mathematical expressions reflect properties about systems and reveal
features that are perhaps not obvious to us at the outset. It will
also reveal some limitations: if we are too simplistic, then the
results will not necessarily behave as we expect. The value of this
kind of exercise is that it helps analysts and researchers to think
and understand.

\begin{enumerate}
\item Show that, if we constrain two variables $q_1$ and $q_2$
so that their product is fixed, we create a {\em dependency}
between the variables. Hint: plot $q_1q_2 = k$, for various values
of $k$ on a plot of $q_1$ versus $q_2$.

\item Consider the following graph of two communicating systems.
\begin{figure}[ht]
\psfig{file=twin.eps,width=5cm}
\end{figure}
The arrows represent work-flow and the arrows indicate the direction
of the flows. The real, positive adjacency matrix of this simple graph is
\beq
A = \left(
\begin{array}{cc}
\alpha_1 & \beta\\
\beta & \alpha_2\\
\end{array}
\right)
\eeq
We define a workload vector $\vec L = \left(\begin{array}{c}
L_1\\L_2
\end{array}\right)$ and the productivity (by analogy with connectivity)
as $P = \vec L^{\rm T} A \vec L$.
Write down the productivity and show that it is a scalar.

\item What is the minimum value of $P$. 
\item What is the maximum value of $P$?
\item Use the method of stationary variation on the productivity to
see what it tells you about the non-trivial solutions for $\vec L$.
Show that non-trivial solutions exist only if $\beta^2 = \alpha_1\alpha_2$.
Bearing in mind that the two nodes are independent of one another,
what is the significance of this solution?

\item Let us now couple to two systems together formally, making them dependent on one another.
Consider now the same system with the additional constraint $L_1 L_2
\le k$, some some constant $k$.  Use the Lagrange method with
multiplier $\lambda$ to maximize
\beq
{\cal P} =  \vec L^{\rm T} A \vec L - \lambda (L_1L_2 - k)
\eeq
and show that this leads to the coupled linear equations
\beq
A  \left(\begin{array}{c}
L_1\\L_2
\end{array}\right) = \lambda \left(\begin{array}{c}
L_2\\L_1
\end{array}\right)
\eeq
Find the values of $\lambda$ for which non-trivial solutions exist for
$L_1,L_2$.  Show that there are two solutions $(L_1,L_2) \propto 
(\pm\alpha_2,\alpha_1)$.

\item Show that the stationary value of the productivity is
\beq
P \propto \beta L_1L_2.
\eeq
Suggest a reason why the result is proportional to $\beta$.  Comment
on the symmetry of this answer with respect to $\alpha_1$ and
$\alpha_2$.


\item Repeat the calculation for a graph in which the work-flow between
the hosts travels only one way, i.e. 
\beq
A = \left(
\begin{array}{cc}
\alpha_1 & \beta\\
0 & \alpha_2\\
\end{array}
\right).
\eeq
Draw the graph of this matrix and comment on the values. What does it
tell you? Does it make sense? The graph is now asymmetrical with
respect to $\alpha_1$ and $\alpha_2$.  Find the stationary value of
the productivity and determine which host drives the productivity? If
you were to spend more money on one of the hosts in this setup, which
one would you choose?

Comment on this analysis. What does the mathematics show you? What are
the limitations of the model.
\end{enumerate}
\end{exercise}
\begin{solution}
\end{solution}



\chapter{Decision theory}


\begin{exercise}
This problem is about the theory of rational decision making.
\begin{enumerate}

\item Explain in your own words what is meant by a mathematical game.

\item Explain what is meant by `payoff' or `utility'. What role do currency systems
play in defining payoff?

\item Explain how a two-person game can be represented as a {\em pair} of matrices.

\item Examine the game in example 169. Explain, with a critical eye, 
how the values are arrived at. If you disagree with the values in this matrix,
explain why and argue for your own values.
\end{enumerate}
\end{exercise}
\begin{solution}
\end{solution}


\begin{exercise}
\begin{enumerate}

\item Use the Gambit game software to analyze the model of software
updating in example 169 in section 19.4. Look for the Nash equilibria
of the game. If you have modified the game, in the previous exercise,
work out the solutions of both versions and compare them.

Note, the Gambit software uses the term `outcomes' for the payoff. You
should choose the normal form for game modelling.

\item Try modifying and extending the model to investigate how stable the
equilibrium points are to changes in the precise values. Explain your
thinking.

\item Evaluate and comment on the usefulness of the procedure you have
just undertaken. How reliable is it in making decisions? How realistic is it
as a model of rational decision making? How might you use it in practice?
\end{enumerate}
\end{exercise}
\begin{solution}
\end{solution}


\begin{exercise}
Think of a decision situation of your own that can be represented as
a two person game and use the Gambit software to find a solution
to the game. Interpret the results -- you will be presenting this
example to the class. 
\end{exercise}
\begin{solution}
\end{solution}



\chapter{Game theory}

\begin{exercise}
These questions are to guide your reading. You do not need to submit them.
\begin{enumerate}

\item What does it mean to say that a game is `zero-sum'?

\item Describe the so-called Prisoner's Dilemma game. Is it a zero-sum game?
Explain your answer.

\item Mention some applications of the game.

\end{enumerate}
\end{exercise}
\begin{solution}
\end{solution}


\end{document}