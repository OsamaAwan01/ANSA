\documentclass{slides}

\title{Week 6: Graphs and diagrams}
\usepackage{psfig}

\begin{document}
\maketitle

\slide{Representing relationships}

Graphs easily represent relationships between discrete
objects.

\begin{itemize}
\item Entity relation diagrams.
\item Hierarchies (trees).
\item Networks.
\end{itemize}

Diagrams have nodes and links. The links can be
directed or undirected.

Diagrams are systems. Freedoms: postion, colour, shape, direction,..
Constraints: size, location, discrete...

\slide{Graph}

A graph is, formally, a pair $(X,\Gamma)$ of nodes or locations $X$
and links (or edges) $\Gamma$, which are mappings from one position to
another: $\Gamma: X \rightarrow X$.

\psfig{file=g1.eps,width=14cm}

The arrows can represent relationships:

\begin{itemize}
\item $A \rightarrow B$: dominates, relies on, promises to, ...
\item $A \leftrightarrow B$ or $A -- B$: is associated with, is correlated with, ...
\end{itemize}

\slide{Degree and adjacency}

The degree $k_i$ of a node $i$ is the number of links is has. For directed
graphs we distinguish in-degree and out-degree:

\psfig{file=degree.eps,width=14cm}

Graphs are easily represented as {\em adjacency matrices}: 1 = link, 0 = no link.
e.g. (first figure)

$$
A = \left(\begin{array}{cccc}
0 & 1 & 0 & 0\\
0 & 0 & 1 & 1\\
0 & 0 & 0 & 0\\
0 & 0 & 0 & 0\\
\end{array}\right)
$$

A is symmetrical if the graph is undirected

\slide{Connectivity}

The proportion of full-connectivity in a graph is $\chi$.
Let $\vec h = \{h_i\}$ be 1 if a node/host is available, else 0.

$$
\chi = \frac{1}{N(N-1)} \vec h^{\rm T} A \vec h
$$

This equals 1 if the graph is {\em complete}, i.e. a link between
every pair of nodes. It is zero if there are no links.

\slide{Graphs as landscapes}

Imagine a 3d view of the graph, with some height function:

\psfig{file=landscape.eps,width=14cm}

Let height represent `importance'. We can measure importance in
different ways. A very useful way is to use {\em eigenvalue centrality}.

\slide{Centrality}

How important is a node?

An important node is connected to many other {\em important} nodes (circular).

$$
I_i \propto \sum_{j \in {\rm neighbours}(i)} I_j
$$
The adjacency matrix is the matrix of neighbours:
$$
I_i \propto \left( \sum_j A_{ij}\right) I_j
$$
Let constant of proportionality be $k = 1/\lambda$ and we have that $\vec I$ is
an eigenvector of $A$:
$$
A\vec I = \lambda \vec I.
$$


\slide{Centrality importance}

The (eigenvector) centrality is used can be used to give a ranking
of node importance in all situations where connectivity is important:

\begin{itemize}
\item Google uses this to rank web pages
\item Vulnerability to virus spreading
\item Resource bottlenecks
\end{itemize}

NB: we want the eigenvector of the largest eigenvalue, since this
is formed from the purely positive sum of neighbours.

\slide{Resource flows}

Directed graphs represent flows, e.g. current packet flow.

Undirected graphs represent equilibrium (balance), e.g. average traffic usage

We can use centrality to find the importance structure (see 11.7.2).

\slide{Percolation}

What if the links are random (e.g. ad hoc nets), or we only know a probability distribution
of the link degrees (e.g. Internet). 
$$
p(k) : \sum_k p(k) = 1
$$

What is the probability of there being at least one path that reaches
every node. 

The percolation transition occurs when:
$$
\langle k\rangle^2 + \sum_k k(k-2) p(k) > \log N
$$
for $N$ nodes. (See 11.6)

\slide{Normalization}

See chap 14. Recall database theory.

If a directed relationship links multi-dimensional data at each node
(i.e. a table of information), we can try to ensure that degrees of
freedom (data objects) only occur a minimum number of times --- to
avoid repetition and inconsistency, loops etc.

\slide{Topology}

Graphs are used to extract a particular discrete view of structural
relationships. They generalize to smooth surfaces with a certain
{\em topology}, in more general systems.

\slide{Applications}

Quality of service in networks - avoiding bottlenecks.

Finding security weaknesses.

Human time management and scheduling: start with
a simple view:

\psfig{file=org1.eps,width=14cm}

\slide{Applications, continued...}

Find out who is the most important, using
a centrality view:

\psfig{file=org2.eps,width=14cm}

\slide{Eigenvvalues in octave/matlab}

\begin{verbatim}
octave:1> A = [ 0 1 0; 1 0 1; 1 1 0 ]
A =

  0  1  0
  1  0  1
  1  1  0

octave:2> [vec,val] = eig(A)
vec =

   -7.5294e-01    4.0045e-01    7.0711e-01
    4.6534e-01    6.4794e-01   -7.0711e-01
    4.6534e-01    6.4794e-01   -4.4981e-17

val =

  -0.61803   0.00000   0.00000
   0.00000   1.61803   0.00000
   0.00000   0.00000  -1.00000

octave:3>
\end{verbatim}
\end{document}