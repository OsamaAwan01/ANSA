\documentclass{slides}

\title{Week 3: From the discrete to the continuous}
\usepackage{psfig}

\begin{document}
\maketitle

Chapters 5 and 8 in the book.

We use two kinds of descriptions of phenomena:

\begin{enumerate}
\item {\bf Continuous}: infinitely detailed, e.g. probabilities $\in [0,1]$.
\item {\bf Discrete}: countable, discrete variables e.g. $q \in\{windows,linux,mac\}$. (NB Discrete $\not =$ discreet!)
\end{enumerate}

Sometimes we use the historical phrases

\begin{enumerate}
\item Analogue.
\item Digital.
\end{enumerate}

The values we measure in an experiment belong to either discrete or continuous sets.

\slide{Measurement}

We associate a value $q \in \{ ... \}$ with a time or a location e.g. $t \in \{....\}$.

Pairs of measurements and times belong together: $(q_1,t_1), (q_2,t_2), ...$.

These form a mapping from time to measured values $q$.

\psfig{file=mapping.eps,width=10cm}

\slide{Mapping}

\begin{enumerate}
\item Continuous: we combine the pairs into a smooth function $(q_1,t_1), (q_2,t_2), ... \rightarrow  q(t)$.

\item Discrete: an example of this is Perl's associative arrays:
\begin{verbatim}
$array{t1} = q1;
$array{t2} = q2;
\end{verbatim}
etc.
$t$ is called a parameter, or the ``address'' of the measured value.
\end{enumerate}

\slide{Sets}

Operations to sub-classify or cross-classify sets, based on Boolean calculus, Venn diagrams:

e.g. intersection between input and output sets: $I \cap O$ (represents transmitted information - what the input and output have in common).

\psfig{file=venn1.eps,width=10cm}

Union $I \cup O$ is the whole region.


\slide{Changes/transitions}

Call an observed value a state (property) of the system. How do states change?

\begin{enumerate}
\item Continuous: this is a great simplification that enables us to see general trends, by
ignoring small details. This was Newton's great achievement -- the continuum approximation:
we can define gradients or derivatives,
$$
\frac{dq}{dt} = \lim{dt\rightarrow 0} \frac{q(t+dt)-q(t)}{dt}.
$$
This gives us simple algebraic forms that are easy to work with and to write down.
(Why keep complicated tables of data if we don't have to?)

\item Discrete: Changes do not always happen in regular patterns at regular intervals, e.g.
random arrivals. Then we have no choice but to look at transition matrices:
$$
(q_1,t_1) \rightarrow  (q_2,t_2)\rightarrow  (q_3,t_3) ...
$$

\end{enumerate}

\slide{Determinism}

The probability of changing from state $q_1$ into state $q_2$ is
$$
P(q_2|q_1) = 
\left(
\begin{array}{cccc}
. & . & . & .\\
. & . & . & .\\
. & . & . & .\\
. & . & . & .\\
\end{array}
\right)
$$
This is a stochastic graph or transition matrix. If the probabilities are all 1 or 0 then
the system is said to be {\em deterministic}, i.e. absolutely predictable.


\slide{Modelling change}



\begin{enumerate}
\item Continuum approximation: if we can simplify and write down a familar function,
e.g. $\sin(\omega t)$ for waves etc, this is very easy to work with. Models are compact and
we have all the tools of continuous mathematics at play with,.

This is usually possible as the numbers of observations become very large. A model is successful
if our fitted function $q(t)$ agrees with observations with only small deviations.

\item Discrete: Small primitive change processes have to be described in `jumps'.
We use graphs or language theory, or automata (state machines).

Specifying discrete transitions is tedious and quickly becomes impractical unless it can
be automated.
\end{enumerate}

\slide{Finite state machines}

Consists of a set of states, an alphabet and a transition matrix (plus some details).
i.e. it is a graph of nodes, and labelled edges.

\slide{Language short cuts}

The regular languages are representable by finite state machines.
We can represent {\em classes} of discrete changes using {\em regular expressions},
e.g. valid TCP sequences:
\begin{verbatim}
  [SYN][ACK][DATA_1][DATA_2]....[FIN][ACK]  
= [SYN][ACK][DATA_1].*[FIN][ACK]
\end{verbatim}
The $\cdot$ means `any symbol' and the Kleene star $*$ means `any number of'.

\end{document}