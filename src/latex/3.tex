\documentclass{slides}

\title{Week 4: Configuration management}
\usepackage{psfig}

\begin{document}
\maketitle

\slide{Apply modelling to a case we know...}

Hardware are software configuration.
\begin{itemize}
\item Set up
\item Maintenance
\end{itemize}

To model this, we use our two approaches:
\begin{itemize}
\item Discrete: each change is an operation that transforms a state $q$ into a new state $q'$, by some transition
matrix.
\item Continuous: look at the long-term trends $q(t)$ and find the rates at which changes happen. What does that mean about the need for management?
\end{itemize}

Changes occur:
\begin{itemize}
\item By us (intentional change)
\item By accident (random error)
\end{itemize}
Later we shall relate changes to information theory in order to discuss reliability.


\slide{Discrete configuration (section 15.2)}

Imagine that we represent file objects (high level) by a vector:
$$
\vec q = \left(
\begin{array}{c}
q_1\\
q_2\\
q_3\\
\end{array}
\right)
$$
where e.g. $q_1$ is a file, $q_2$ is a process, $q_3$ is an access permission etc.
We can make any abstraction we like.

To make a change $q\rightarrow q'$, we need a {\em matrix} or operator.
This is just one representation of the problem. There might be others.
Actually we don't {\em need} a representation, it is enough to use algebra:
$$
\vec q_{fixed} = \hat O\,\vec q_{broken}
$$
i.e.

$$
\left(
\begin{array}{c}
q_1'\\
q_2'\\
q_3'\\
\end{array}
\right) =
\left(
\begin{array}{ccc}
? & ? & ?\\
? & ? & ?\\
? & ? & ?\\
\end{array}
\right)
\left(
\begin{array}{c}
q_1\\
q_2\\
q_3\\
\end{array}
\right)
$$

Perhaps we can think of other representations. Matrices are easy to work with,
provided the number of rows is small!

The power is that we have a way of proving things on paper.

\slide{Operators}

Each change or repair is an operation mediated by operator $\hat O$.

Suppose we start with a system `out of the box', in state $q_0$. We
decide what we want and then set this up by performing a string of
operations on it, in order to obtain a policy state.
$$
\vec q_{policy} = \hat O_n ... \hat O_3\hat O_2\hat O_1\; \vec q_0.
$$
or, equivalently
$$
\vec q_{policy}^{\rm T} = \vec q_0^{\rm T} \hat O_1^{\rm T} \hat O_2^{\rm T} ...\hat O_n^{\rm T}.
$$
e.g. $\hat O_1$ is edit /etc/passwd, $O_2$ is `install software', etc..

This notation allows us to discuss the algebra of changes.

\slide{Ordering}

We have to be careful. Do the operators commute? (Is order important?)
$$
[\hat O_1,\hat O_2] = \hat O_1\hat O_2 - \hat O_2\hat O_1 = ?
$$
If this is zero, order does not matter. If non-zero, we might get the wrong
answer unless the order is guaranteed.

e.g. The cfengine philsophy tries to provide operators that commute. The IsConf
philosophy says, reset the machine back to $q_0$ and remember the order. These
two approaches are not as different as they sound.

\slide{Convergence}

To automate the configuration maintenance we would like to be able to
apply `dumb operations' $\hat C$ repeatedly. This means we place additional
restrictions on the operators. 

Immunity model says: create an operator that takes us from any state to our
policy state, and no further! i.e. if we apply operator to a non-policy state,
we get the policy state. If we apply it to the policy state, nothing happens.

$$
\hat C \vec q_{broken} = \vec q_{policy}
$$

$$
\hat C \vec q_{policy} = \vec q_{policy}
$$

This is called convergence.

e.g. $\hat C$ is "add mark to passwd file if user mark does not already exist".

What is new is that the operators need to ``observe'' the state of the system
before acting.

\slide{Continuous maintenance (chapter 16)}

Make general statements about the rate of damage vs the rate of repair.
How quickly do we have to act to maintain trends in behaviour, i.e. policy?

We use a {\em local averaging procedure} to find a continuum picture.
This only works for numerical variables!

Divide time up into regions and use average value in a region as the trend. 
Interpolate a smooth curve through these (see fig. 16.1). This is the picture
we see approximately when we zoom out and look at long times.
i.e. we approximate the complex form by a much simpler form --  we suppress
details in order to achieve clarity!

A change in each region of the system can be written relative to the local mean value
$\langle q\rangle_t$:
$$
q(t) = \langle q\rangle_t + \delta q
$$
i.e. signal = trend + flucuation.

\slide{The maintenance theorem}

This is a complicated argument to say something simple: we want to manage
systems based on their long-term trends, not on micromanaging every fluctuation.
(Micro-management requires system redesign, we can not call it ``management''!)

{\em Question: can we apply this to manage systems on Mars, where the communication interval is about 5-20 minutes?}

The theorem says that we should associate policy with $ \langle q\rangle_t$, i.e.
the average state. We should allow some `slack' into the policy because we cannot
define the precise state realistically. So we define policy as the acceptable
trends in sytsem development, which means we allow a slack of the order $\delta q$.

Now, to ensure that the system does not deviate by more than $\delta q$, we need to
repair the change within a time $\Delta t$, where:
$$
\delta q = \frac{dq}{dt}\delta t \ll q
$$
e.g. security policy might be adapting slowly over weeks (small $dq/dt$), but a
rapid change $\delta q$ should be repaired quickly to maintain the integrity
of the policy.

We can formalize these arguments and say something about the how hard we have to
work to keep a system within certain bounds.

The continuum picture is a picture of fuzzy regulation (this is what we really
mean by management) not of crystalline perfection.

\end{document}