
\documentclass{beamer}

\newtheorem{myrule}{Rule}

\def\beq{\begin{eqnarray}}
\def\eeq{\end{eqnarray}}
\def\2{\frac{1}{2}}



\mode<presentation>
{
 \usetheme{Warsaw}
%\usecolortheme[rgb={0,0.7,0.1}]{structure}
%  \usetheme{default}
%  \usetheme{Boadilla}
%  \usetheme{Madrid}
%  \usetheme{Pittsburg}
%  \usetheme{Boadilla}

  % or ...


  \setbeamercovered{transparent}
  % or whatever (possibly just delete it)
}


\usepackage{epsfig}
\newtheorem{axiom}{Axiom}
\usepackage[english]{babel}
\usepackage[latin1]{inputenc}
\usepackage{times}
\usepackage[T1]{fontenc}
% Or whatever. Note that the encoding and the font should match. If T1
% does not look nice, try deleting the line with the fontenc.

\title[Game Theory...] % (optional, use only with long paper titles)
{Rational Decisions}


\author[] % (optional, use only with lots of authors)
{MB}
% - Give the names in the same order as the appear in the paper.
% - Use the \inst{?} command only if the authors have different
%   affiliation.

\institute[Oslo University College] % (optional, but mostly needed)
{
  Department of Computer Science\\
  Oslo Univeristy College
% - Use the \inst command only if there are several affiliations.
% - Keep it simple, no one is interested in your street address.
}

\date[MS007A] % (optional, should be abbreviation of conference name)
{November 2007}

\subject{Rational Decisions}
% This is only inserted into the PDF information catalog. Can be left
% out. 
%\usecolortheme{beetle}

\begin{document}

\begin{frame}
  \titlepage
\end{frame}


\begin{frame}
\frametitle{Rational decision}

\begin{itemize}
\item Chapter 19.\vspace{0.2cm}

\item How to make rational decisions, i.e. how to justify conclusions
by formalizing questions and answers.\vspace{0.2cm}

\item Optimization of goals described by policy.\vspace{0.2cm}

\item We must normally specify 
policy to eliminate arbitrariness in decisions.\vspace{0.2cm}

\item Two steps:\vspace{0.2cm}
\begin{itemize}
\item Use models built so far to understand how processes occur
\item Add a {\bf value judgement} (function) about the behaviour
\end{itemize}
\end{itemize}
\end{frame}

%%

\begin{frame}
\frametitle{Classification}

\begin{itemize}
\item Decisions require us to categorize or classify options (Bacon / ITIL). Set classification
like in cfengine.\vspace{0.2cm}

{\tt\small
linux|solaris::\\
~\\
~~~~~   rule...}

\item Decisions can involve conflicting interests. How do these compete?
How do we resolve the competition rationally?\vspace{0.2cm}

\item Simplest decisions are discrete yes/no Boolean
decisions (AND/NOT/OR). We know these from programming.\vspace{0.2cm}

\item Continuous, value-based decisions must be decided:\vspace{0.2cm}
\begin{itemize}
\item Value / currency based.
\item Payoff, utility, outcome
\item Algebra of optimization
\end{itemize}
\end{itemize}

\end{frame}

%%

\begin{frame}\frametitle{Multiple possibilities}

\begin{itemize}
\item ``Conflicts of interest''\vspace{0.3cm}

\item What if more than one ``right'' answer?\vspace{0.2cm}

\item Weight or rank the importance of the possibilities using expected
payoff (as in graphs).\vspace{0.2cm}

\item Might end up with a single answer or several equally good answers.\vspace{0.2cm}

\begin{itemize}
\item Choose one by policy?
\item Choose a mixture?
\end{itemize}

\item We can deal with these issues using {\bf game theory}
\end{itemize}
\end{frame}

%%

\begin{frame}\frametitle{Games}

\begin{itemize}
\item A game is a competition between $N$ players.\vspace{0.2cm}

\item The theory of games is about competition between different goals or interests
for a common resource. e.g. two player games like chess.\vspace{0.2cm}

\begin{itemize}
\item Goals or interests become ``players''.
\item Player N wins higher score the closer (s)he gets to policy.\vspace{0.2cm}
\end{itemize}
\item We shall focus on 2-player games. This is hard enough.\vspace{0.2cm}

\item With a little imagination, we can go a long way with 2 player games.\vspace{0.2cm}

\item As with all models, there are discrete and continuous forms.
\end{itemize}
\end{frame}

%%

\begin{frame}\frametitle{Discrete, extensive form of games}

\begin{itemize}
\item Make a {\bf decision tree} of every possible action for every player at every move.

\begin{itemize}
\item Very complex (micromanagement).

\item Add up the payoffs for each path through the game to find the best compromises,
given that each player is trying to do the same.
\item Advantage is we can retain the order of decisions.
\end{itemize}

\item Disadvantages
\begin{itemize}
\item Complex model. 
\item Expensive to make. 
\item More details than we know?\vspace{0.2cm}
\end{itemize}
\end{itemize}
\end{frame}

\begin{frame}
\frametitle{Extensive form of game}
\epsfig{file=extensive.eps,width=10cm}
\end{frame}

%%

\begin{frame}
\frametitle{Continuum (averaged) strategic form}

\begin{itemize}
\item Also called the {\bf normal} form.\vspace{0.2cm}

\item Like a ``spreadsheet'' for valuating competing alternatives\vspace{0.2cm}

\item Deals only with the main branches of the tree.
(This is most useful for us)\vspace{0.2cm}

\item Lump together actions that form single pathways into {\em classes} of play-behaviour.\vspace{0.2cm}

\item High level management (cheap).\vspace{0.2cm}

\item Order of decisions is lost -- both players are considered to act at the same
time (like a duel).
\end{itemize}
\end{frame}

%%

\begin{frame}\frametitle{Payoff matrix $\Pi$ -- strategic form}

\begin{itemize}
\item Matrix of possible payoffs for each player, given different combinations of 
stategy decisions.
\end{itemize}

\psfig{file=game.eps,width=4cm}


\begin{itemize}
\item Player 1: $A,B,C$
\item Player 2: $\alpha,\beta,\gamma$

\item Players can also use ``mixed strategies''

 $c_1A+c_2B+c_3C+...$ etc

\item Could be mixed during one game, or over several ``plays''.
\end{itemize}
\end{frame}

%%

\begin{frame}\frametitle{Example game matrix}

\begin{center}
\rm
\begin{tabular}{l|c|c|c|}
 & Security & Bug in   & Missing \\
 & hole     & function & function\\
\hline
{\small Upgrade version now} & (10,5) & (10,0) & (5,-5)\\
{\small Test then upgrade}   & (5,5)  & (3,9)  & (0,8)\\
{\small Keep parallel versions}   & (-10,5)& (-1,10) & (0,10)\\
\end{tabular}
\end{center}


Payoffs need to be chosen ``rationally''. This is the crux.
Can often be subjective, so be careful.
\end{frame}

%%


\begin{frame}\frametitle{Making the optimal decision}

\begin{itemize}
\item A ``solution'' of the game is a specification of optimal
strategies for each player and the value received by
each player when they behave rationally.\vspace{0.2cm}

\item Both players are trying to maximize their payoff at the same
time.\vspace{0.2cm}

\item Saddle-points and equilibria for zero-sum.
\end{itemize}
\end{frame}

%%

\begin{frame}
\frametitle{Solution methods}

\begin{itemize}
\item If $\Pi_1+\Pi_2 = 0$ (zero sum game), the game is solved by ``minimax''.\vspace{0.2cm}

\item There is a ``best answer'' if the game has a saddle-point, otherwise there
is only a combination of answers that gives best result.\vspace{0.2cm}

\item If $\Pi_1+\Pi_2 = [{\rm const}_{ij}]$ have to use more general equilibrium
ideas -- e.g. Nash equilibrium.
\end{itemize}
\end{frame}



\begin{frame}
\frametitle{Summary}

\begin{itemize}
\item Formulate a decision in terms of some scale that can be measured or modelled\vspace{0.2cm}

\item Try to formulate a policy for judging the value of the answer to some goal\vspace{0.2cm}

\item Maximize the value of the outcome, possibly with competing interests\vspace{0.2cm}

\item Find stable balance points between all interests (the solution)
\end{itemize}
\end{frame}



\end{document}